\documentclass[12pt]{article}
\usepackage{latexsym}
\usepackage{tikz}
\usepackage{tkz-graph}
\usetikzlibrary{shapes}

\parskip=3pt

\setlength{\textheight}{8.5in}
\setlength{\textwidth}{6in}
\setlength{\topmargin}{0in}
\setlength{\oddsidemargin}{0in}
\setlength{\evensidemargin}{0in}

\newtheorem{Theorem}{Theorem}[section]
\newtheorem{Corollary}[Theorem]{Corollary}
\newtheorem{Lemma}[Theorem]{Lemma}
\newtheorem{Observation}[Theorem]{Observation}

\def\lc{\left\lceil}   
\def\rc{\right\rceil}
\def\inst#1{$^{#1}$}
\def\DCD{($diamond$, $co$-$diamond$)}

\title{On k-critical {\DCD}-free graphs}

\author{
	Dallas J. Fraser\inst{1}
	\and Ang\`ele M. Hamel'\inst{1}
	\and Ch\'inh T. Ho\`ang\inst{1}
}
\begin{document}
\maketitle

\begin{center}
{\footnotesize

\inst{1}, Department of Physics and Computer Science, Wilfrid Laurier
University, \\Waterloo, Ontario, Canada}

\end{center}

\begin{abstract}
A graph is $k$-critical if it is $k$-chromatic but each of its proper induced subgraphs is ($k-1$)-colorable. We show that the set of $k$-critical {\DCD}-free for all $k$. Our result implies the existence of a certifying algorithm for $k$-coloring {\DCD}-free graphs.

\noindent{\em Keywords}: Graph coloring, $diamond$, $co$-$diamond$
\end{abstract}


\section{Introduction}\label{sec:intro}

Graph coloring is a well-studied problem in comptuer science and discrete mathematics. Determining the chromatic number of a grpah is a NP-hard problem. But for many classes of graphs, such as perfect graphs, the problem can be solved in polynomial time. In 2002 \cite{BraFud2002}, research was done on prime {\DCD}-free graphs and provided a starting point for this research.

The point of view in this article is motivated by the idea of a ``certifying algorithm''. An algorithm is {\it certifying} if it returns with each output a simple and easily verifiable certificate that the particular output is correct. For example, a certifying algorithm for the bipartite grpah recognition would return either a 2-coloring of the input graph, thus providing that it is bipartite, or an odd cycle, thus proving it is not bipartite. A certifying algorithm for planarity would return either an embedding of the graph in a plane, or one of the two Kuratowski subgraphs proving the input graph is not planar.

A grpah is $k$-critical if it is $k$-chromatic but each of its proper induced subgraphs is $(k-1)$-colorbale. Here we prove that the number of $k$-critical {\DCD}-free graphs is finite for every fixed $k$. In section~\ref{sec:definitions}, we give definitions and background on our problem. In section\ref{sec:updates}, we expand upoon previous Lemmas \cite{BraFud2002}. In section~\ref{sec:characterization}, we give the proof of our main results.

\section{Definitions and background}\label{sec:definitions}
A $k$-coloring of a graph $G=(V,E)$ is a mapping $f: V \rightarrow \{1,\ldots, k\}$ such that $f(u) \not= f(v)$ whenever $uv \in E$. Given a coloring, a {\it color class} is the set of all vertices of the same color.  The chromatic number $\chi(G)$ of a graph $G$ is the smallest $k$ such that $G$ is $k$-colorable.  $G$ is $k$-chromatic if $\chi(G) = k$.  A graph $G$ is {\em $k$-critical} if it is $k$-chromatic and none of its proper induced subgraphs is $k$-chromatic (that is, all of its proper induced subgraphs are $(k-1)$-colorable).  We say that a graph is {\it critical} if it is $k$-critical for some $k$.  Let $N(v)$ be the set of neighbors of $v$. A vertex of $G$ is {\it universal} if it is adjacent to every other vertex of $G$. Vertices $u,v$ are comparable if $N(u) \subseteq N(v)$, or vice versa. If $X$ is a set of vertices of $G$, then $G[X]$ denotes the subgraph if $G$ induced by $X$.  A set $A$ of vertices is {\it complete} to a set $B$ of vertices if there are all edges between $A$ and $B$.  Given two graphs $G$ and $ H$, the graph $F$ is the {\it join} of $G$ and $H$ if $F$ is obtained by taking $G$ and $H$ and joining every vertex in $G$ to every vertex in $H$ by an edge.  As usual, $K_t$ denotes the clique on $t$ vertices; and $C_t$ denotes the induced cycle on $t$ vertices. The complement of $G$ is denoted by $\overline{G}$. 

A $diamond$ refers to a graph G with 4 vertices such that two vertices are universal. A $co$-$diamon$ refers to a graph $G$ which is the complement of a $diamond$. For $U \subseteq V$ let $G(U)$ denote the subgraph of $G$ induced by $U$. Throughout this paper, all subgrpahs are understood to be induced. If $H$ is a subgraph of F then a vertex $v$ not in $H$ is called a $k-vertex$ for $H$ if $v$ has exactly $k$ neighbors in $H$. For a vertex set $U$ in $H$ with $|U| = k$, let $N_U$ denote the set of $k-vertices$ for $H$ being adjacent to all vertices in $U$.  The following lemmas have been previously established.

\begin{Theorem}\label{thm:strong-perfect}
(odd-hole, odd-anti-hole)-free graphs are perfect. $\Box$
\end{Theorem}

\begin{Lemma}\label{lem:complement-k-vertex}
If $G$ with $v$ vertices has no $k$-vertex then $\overline{G}$ has no ($v-k$)-vertex. $\Box$
\end{Lemma}
\begin{Lemma}\label{lem:c5-9v}{\rm \cite{BraFud2002}}
If $G$ contains a $C_L$ then $G$ has most nine vertices. $\Box$
\end{Lemma}

\section{Updates to previous Lemmas}\label{sec:updates}
In this section, we prove 3 properties of {\DCD}-free graphs.
\begin{Lemma}\label{lem:c7-free}
$G$ cannot contain an induced $C_l$, $l \geq 7$.
\end{Lemma}
\noindent {\it Proof}. Let $G$ be a $C_l$, $l \geq 7$ with vertices 1, 2, ...,l-1, l. This forms a $co$-$diamond$ with (1,2) 4, l-1 and $G$ is {\DCD}-free. $\Box$
\begin{Lemma}\label{lem:co-c7-free}
$G$ cannot contain an induced $\overline{C_l}$, $l \geq 7$
\end{Lemma}
\noindent{\it Proof}. Let $G$ be a $\overline{C_l}$, $l \geq 7$ with vertices 1, 2, ...,l-1, l. This forms a $diamond$ with (1,2, 4 , l-1) and $G$ is {\DCD}-free. $\Box$

\section{The structure of $k$-critical {\DCD}-free graphs}\label{sec:characterization}
In this section, we show tat $\mathcal{C}_k$ for a fixed $k > 3$ is finite and contain only:
\begin{itemize}
\item[(i)]
$K_n$ where $n = k$
\end{itemize}
\noindent {\it Proof}. If $G$ contains a $C_5$ then it can only b 3-critical since it can have at most 9 vertices and all its variations are 3-colorable. If $G$ does not contain a $C_5$ then by Lemmas \ref{lem:c7-free},\ref{lem:co-c7-free}, and \ref{thm:strong-perfect} $G$ is perfect and can only be $k$-critical if $G$ is a clique with $k$ vertices

\begin{center}
{\bf Acknowledgement}
\end{center}
This work was done by authors  Laurier University. The authors A.M.H. and C.T.H. were each supported by individual NSERC Discovery Grants. D.J.F was supported by an NSERC Undergraduate Student Research Award.


\clearpage
\begin{thebibliography}{99}

\bibitem{BraFud2002}
  A.~Brandstadt and S.~Mahfud, Maximum Weight Stable Set on graphs without $claw$ and $co$-$claw$ can be solved in linear time, manuscript.

\bibitem{DhaHam2014}
  H. Dhaliwal, A. Hamel, C. Ho\'{a}ng, F. Maffray, T. McConnel and S. Panait, On Color-critical ($P_5, \overline{P}_5$)-free graphs, manuscript.

\end{thebibliography}

\end{document}
