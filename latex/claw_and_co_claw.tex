\documentclass[12pt]{article}
\usepackage{latexsym}
\usepackage{tikz}
\usepackage{tkz-graph}
\usetikzlibrary{shapes}

\parskip=3pt

\setlength{\textheight}{8.5in}
\setlength{\textwidth}{6in}
\setlength{\topmargin}{0in}
\setlength{\oddsidemargin}{0in}
\setlength{\evensidemargin}{0in}

\newtheorem{Theorem}{Theorem}[section]
\newtheorem{Corollary}[Theorem]{Corollary}
\newtheorem{Lemma}[Theorem]{Lemma}
\newtheorem{Observation}[Theorem]{Observation}

\def\lc{\left\lceil}   
\def\rc{\right\rceil}
\def\inst#1{$^{#1}$}
\def\CCO{($claw$, $co$-$claw$)}

\title{On k-critical ($claw$, $co$-$claw$)-free graphs}

\author{
  Dallas J. Fraser\inst{1}
  \and Ang\`ele M. Hamel\inst{1}
  \and Ch\'inh T. Ho\`ang\inst{1}
}
\begin{document}
\maketitle

\begin{center}
{\footnotesize

\inst{1} Department of Physics and Computer Science, Wilfrid Laurier
University, \\ Waterloo, Ontario, Canada}
%\\ \texttt{choang@wlu.ca}


\end{center}
%
\begin{abstract}
A graph is $k$-critical if it is $k$-chromatic but each of its proper
induced subgraphs is ($k-1$)-colorable. There is an infinite number of $k$-critical {\CCO}-free graphs when $k = 3$ since all odd-cycles are {\CCO}-free. We show that the set of $k$-critical {\CCO}-free graphs for some $k > 3$ is  finite. Our result implies the existence of a certifying algorithm for $k$-coloring {\CCO}-free graphs. 

\noindent{\em Keywords}: Graph coloring, $claw$-free graphs, $co$-$claw$-free graphs
\end{abstract}


\section{Introduction}\label{sec:intro}

Graph coloring is a well-studied problem in computer science and
discrete mathematics.  Determining the chromatic number of a graph
is a NP-hard problem.  But for many classes of graphs, such as
perfect graphs, the problem can be solved in polynomial time.
In 2002 \cite{BraFud2002}, research was done on  prime {\CCO}-free graphs and provided a starting point for this research.

The point of view in this article is motivated by the idea of a ``certifying algorithm''.  An algorithm is {\it certifying} if it returns with each output a simple and easily verifiable certificate that the particular output is correct.  For example, a certifying algorithm for the bipartite graph recognition would return either a 2-coloring of the input graph, thus proving that it is bipartite, or an odd cycle, thus proving it is not bipartite.  A certifying algorithm for planarity would return either an embedding of the graph in a plane, or one of the two Kuratowski subgraphs proving the input graph is not planar.

A graph is $k$-critical if it is $k$-chromatic but each of its proper induced subgraphs is $(k-1)$-colorable. Here we prove that the number of $k$-critical {\CCO}-free graphs is finite for every fixed $k$ $ > 3$.
In section~\ref{sec:definitions}, we give definitions and background on our problem. In section~\ref{sec:updates}, we expand upon previous Lemmas \cite{BraFud2002}. In section~\ref{sec:characterization} , we give the proof of our main results. 

\section{Definitions and background}\label{sec:definitions} 
A $k$-coloring of a graph $G=(V,E)$ is a mapping $f: V \rightarrow \{1,\ldots, k\}$ such that $f(u) \not= f(v)$ whenever $uv \in E$. Given a coloring, a {\it color class} is the set of all vertices of the same color.  The chromatic number $\chi(G)$ of a graph $G$ is the smallest $k$ such that $G$ is $k$-colorable.  $G$ is $k$-chromatic if $\chi(G) = k$.  A graph $G$ is {\em $k$-critical} if it is $k$-chromatic and none of its proper induced subgraphs is $k$-chromatic (that is, all of its proper induced subgraphs are $(k-1)$-colorable).  We say that a graph is {\it critical} if it is $k$-critical for some $k$.  Let $N(v)$ be the set of neighbors of $v$. A vertex of $G$ is {\it universal} if it is adjacent to every other vertex of $G$. Vertices $u,v$ are comparable if $N(u) \subseteq N(v)$, or vice versa. If $X$ is a set of vertices of $G$, then $G[X]$ denotes the subgraph if $G$ induced by $X$.  A set $A$ of vertices is {\it complete} to a set $B$ of vertices if there are all edges between $A$ and $B$.  Given two graphs $G$ and $ H$, the graph $F$ is the {\it join} of $G$ and $H$ if $F$ is obtained by taking $G$ and $H$ and joining every vertex in $G$ to every vertex in $H$ by an edge.  As usual, $K_t$ denotes the clique on $t$ vertices; and $C_t$ denotes the induced cycle on $t$ vertices. The complement of $G$ is denoted by $\overline{G}$. 

A $claw$ refers to a graph G with vertices with one vertex $X$ is universal to {$y_1,y_2,y_3$} which is a stable set. A $co$-$claw$ refers to a graph $G$ which is the complement of a $claw$. For $U \subseteq V$ let $G(U)$ denote the subgraph of $G$ induced by $U$. Throughout this paper, all subgrpahs are understood to be induced. If $H$ is a subgraph of F then a vertex $v$ not in $H$ is called a {\em $k$-vertex} for $H$ if $v$ has exactly $k$ neighbors in $H$. For a vertex set $U$ in $H$ with $|U| = k$, let $N_U$ denote the set of {\em $k$-vertices} for $H$ being adjacent to all vertices in $U$.  The following lemmas have been previously established.

\begin{Theorem}\label{thm:StrongPerfect}
(odd-hole, odd-anti-hole)-free graphs are perfect. $\Box$
\end{Theorem}

\begin{Lemma}\label{lem:connected}
A critical graph is connected. $\Box$
\end{Lemma}

\begin{Lemma}\label{lem:join-critical}{\rm \cite{DhaHam2014}}
Let $G=(V,E)$ be any graph.  Suppose that $V$ admits a partition into
two non-empty sets $V_1$ and $V_2$ such that $V_1$ is complete to
$V_2$.  Then $G$ is critical if and only if the two graphs $G[V_1]$
and $G[V_2]$ are critical. $\Box$
\end{Lemma}

\begin{Lemma}\label{lem:anti-hole-critical}
Let $G$ be a $\overline{C_\ell,}\; \ell >3$ then $G$ is $\lceil l/2 \rceil$-critical. $\Box$
\end{Lemma}

\begin{Lemma}\label{lem:complement-k-vertex}
If $G$ with $v$ vertices has no $k$-vertex then $\overline{G}$ has no ($v-k$)-vertex. $\Box$
\end{Lemma}

\section{Updates to previous Lemmas}\label{sec:updates}
In this section, we prove four properties of {\CCO}-free graphs.

\begin{Lemma}\label{lem:C7Cycle}
If $G$ is a graph which contains an induced $C_\ell,\; \ell \geq 7$, then $G$ itself is such a cycle or is a disconnected graph.
\end{Lemma}
\noindent {\it Proof}.  Expanding \cite{BraFud2002} upon Theorem 2 Claim 1 where it was shown $C_\ell,\; \ell \geq 7$ has no $k$-vertex for $k \geq 1$. If the set of $0$-vertex $= \phi$ then $G$ is a $C_\ell$, else it is a disconnected graph.  $\Box$

\medskip

\begin{Lemma}\label{lem:C69V}
If $G$ contains a $C_6$ then $G$ has at most 9 vertices or is disconnected.
\end{Lemma}
\noindent{\it Proof.} Expanding \cite{BraFud2002} upon Theorem 2 Claim 2 where it was shown $C_6$ has no $k$-vertex for $k \in {1, 2, 3, 5, 6}$. Let $A$ = $N_{1,2,4,5}$, $A$ = $N_{2,35,6}$ and $C$ = $N_{3,4,6,1}$ be a partition of the set of 4-vertices since G is $co$-$claw$. There can be no edge $AB$ ($AB,\;A4,\;A1$), $AC$ ($AC,\;A2,\;A5$), $BC$ ($BC,\;B2,\;B5$) since $G$ is claw-free. The sets can have at most one vertex. Let $A$ contains two vertices $x$ and $y$. If $xy \in E$ then there is a $co$-$claw$ ($xy$, $y1$, $1x$, $3$) but if $xy \not\in E$ then there is a $claw$ ($1x$, $1y$, $16$) which is a contradiction. The same argument applies to $B$ and $C$. There are no $0$-vertex adjacent to a $4$-vertex or $3$-vertex since $G$ is $claw$-free. If the set of $0$-vertex $= \phi$ then $G$ is a $C_6$ with at most 9 vertices, otherwise it is a disconnected graph.  $\Box$



\medskip

\begin{Lemma}\label{lem:c5join}
If $G$ contains a $C_5$ then:
\begin{itemize}
\item[(i)]
$G$ is a $C_5$
\item[(i)]
$G$ is the join of a $C_5$ and $H$
\item[(i)]
$G$ is a disconnected graph
\end{itemize}
\end{Lemma}
\noindent{\it Proof.} Expanding \cite{BraFud2002} Theorem 2 Claim 2 where it was shown $C_5$ has no $k$-vertex for $k \in {1,2,3,4}$ and no 0-vertex adjacent to a 5-vertex. The set of 5-vertex are complete to
 which means that the set of 5-vertices is complete to $C$. Let $A$ be the set of 5-vertex and $B$ be the set of 0-vertex. If $A \neq \phi$ and $B \neq \phi$ then there would be $co$-$claw$. If $A \neq \phi$ then $G$ is a {\it join} of $C$ and $H$ where $H = A$. If $B$ $\neq \phi$ then  $G$ is a disconnected graph. If $A$ $\cup$ $B$ $= \phi$ then $G$ is a $C_5$.  $\Box$

Since $C_5$ is complemetary this Lemma applies for $\overline{C_5}$.
\medskip

\begin{Lemma}\label{lem:co-cl}
If $G$ contains a $\overline{C_\ell}$ where $l \geq 7$ then:
\begin{itemize}
\item[(i)]
$G$ is a $\overline{C_l}$
\item[(i)]
$G$ is the join of a $\overline{C_l}$ and $H$
\end{itemize}
\end{Lemma}
\noindent {\it Proof.} Lemma \ref{lem:C7Cycle} showed all $G$ containing a $C_l$ has no $k$-vertex for $k \in {1, 2, 3, 4, 5, 6}$. So $\overline{C_\ell}$ has no $k$-vertex for $k \in {0, 1, 2, 3, 4, 5,}$ by Lemma \ref{lem:complement-k-vertex} . Let $H$ be the set of 6-vertex. Hence if $H = \phi$, then $G$ is a $\overline{C_\ell}$, otherwise $G$ is a join of $\overline{C_\ell} \cup H$.
\section{The structure of $k$-critical {\CCO}-free graphs}\label{sec:characterization}
%
Let $\mathcal{C}_k$ be the family of $k$-critical {\CCO}-free graphs. Clearly $\mathcal{C}_1 = {K_1}$ and $\mathcal{C}_2 = {K_2}$.
In this section, we show that $\mathcal{C}_k$ for $k > 3$ is finite and is one of the following cases:
\begin{itemize}
\item[(i)]
$\overline{C_l}$, $l = 2k-1$
\item[(ii)]
join of $\overline{C_\ell}$ and $H$ where $H$ is a $k_2$-critical graph and not a $C_\ell,\; \ell > 5$, such that  $\lceil l/2 \rceil + k_2 = k$.
\item[(iii)]
is a clique $K_k$
\end{itemize} 
\noindent {\it Proof.} If $G$ contains an $\overline{C_\ell}$ (odd-anti-hole) then by Lemmas \ref{lem:co-cl} and \ref{lem:c5join} $G$ is a $\overline{C_\ell}$ or a $\overline{C_\ell}$ with $H$. By Lemma \ref{lem:join-critical} $G$ can only be $k$-critical if $H$ is $k_2$-critial where $k_2 + \lceil + l/2 \rceil = k$ since by Lemma \ref{lem:anti-hole-critical} $\overline{C_\ell}$ is $\lceil + l/2 \rceil$-critical. If $G$ contains a $\overline{C_5}$ and is disconnected then by Lemma \ref{lem:connected} it cannot be critical. Note $C_5$ is self-complemantary and itself is a odd-anti-hole. If $G$ contains $C_\ell,\; \ell >5$ and $\ell$ is odd (odd-hole) then by Lemma \ref{lem:C7Cycle} $G$ is an $C_\ell$. Odd-holes are only 3-critical and make $\mathcal{C}_3$ infinite since odd-hole class is infinite. Note an odd-anti-hole cannot be joined with an odd-hole since odd-hole have no $k$-vertex. If $G$ contains no odd-anti-hole and no odd-hole then by Theorem \ref{thm:StrongPerfect} $G$ is perfect and a perfect graph can only be $k$-crtical if $G$ is a clique with $k$ vertices.

\begin{center}
{\bf Acknowledgement}
\end{center}
The authors A.M.H. and C.T.H. were each supported by individual NSERC Discovery Grants. D.J.F was supported by an NSERC Undergraduate Student Research Award.


\clearpage
\begin{thebibliography}{99}

\bibitem{BraFud2002}
  A.~Brandstadt and S.~Mahfud, Maximum Weight Stable Set on graphs without $claw$ and $co$-$claw$ can be solved in linear time, manuscript.

\bibitem{DhaHam2014}
  H. Dhaliwal, A. Hamel, C. Ho\'{a}ng, F. Maffray, T. McConnel and S. Panait, On Color-critical ($P_5, \overline{P}_5$)-free graphs, manuscript.

\end{thebibliography}

\end{document}
