\documentclass[12pt]{article}
\usepackage{latexsym}
\usepackage{tikz}
\usepackage{tkz-graph}
\usepackage{tikz}
\newcommand*\circled[1]{\tikz[baseline=(char.base)]{
            \node[shape=circle,draw,inner sep=2pt] (char) {#1};}}
\usetikzlibrary{shapes}

\parskip=3pt

\setlength{\textheight}{8.5in}
\setlength{\textwidth}{6in}
\setlength{\topmargin}{0in}
\setlength{\oddsidemargin}{0in}
\setlength{\evensidemargin}{0in}

\newtheorem{Theorem}{Theorem}[section]
\newtheorem{Corollary}[Theorem]{Corollary}
\newtheorem{Lemma}[Theorem]{Lemma}
\newtheorem{Observation}[Theorem]{Observation}

\def\lc{\left\lceil}   
\def\rc{\right\rceil}
\def\inst#1{$^{#1}$}
\def\CCD{($claw$, $C_4$, $co$-$diamond$)}

\title{Adding $2$-vertex to $C_5$ {\CCD}-free graphs}

\author{
	Dallas J. Fraser\inst{1}
	\and Ang\`ele M. Hamel'\inst{1}
	\and Ch\'inh T. Ho\`ang\inst{1}
}
\begin{document}
\maketitle

\begin{center}
{\footnotesize

\inst{1}, Department of Physics and Computer Science, Wilfrid Laurier
University, \\Waterloo, Ontario, Canada}

\end{center}

\begin{abstract}
Can color a $C_5$ of a {\CCD}-free graph with a $2$-vertex.

\noindent{\em Keywords}: Graph coloring, $claw$, $C_4$, $co$-$diamond$
\end{abstract}

\section{Oberservations}\label{sec:observations}
In this section, we conclude some observations used to prove that adding a $2$-vertex will never make it critical For the following assume $G$ is {\CCD|-free. Let $X_i$ denote the set of $2$-vertices for $C$ adjacent to $i, i+1$, $Y_i$ denote the the set of $3$-vertices for $C_5$ adjacent to $i-1, i, i+1$ and $U$ denote the set of $5$-vertices for $C_5$.

\begin{Lemma}\label{lem:c5-k-vertex}
$G$ which contains a $C_5$ has no $k$-vertex for $k \in {0, 1, 2, 4}$
\end{Lemma}

\begin{Lemma}\label{lem:max-2-xi}
$|X_i| \leq 1$
\end{Lemma}

\begin{Lemma}\label{lem:xi-no-xi2}
If $X_i \neq \phi$ then $X_{i+2} \cup X_{i+3} = \phi$
\end{Lemma}

\begin{Lemma}\label{lem:max-2-2K}
$|X_0 \cup X_1 \cup X_2 \cup X_3 \cup X_4| \leq 2$
\end{Lemma}

\begin{Lemma}\label{lem:2k-join-3k}
$X_i \;\circled{1}\; Y_{i-1} \cup Y_{i} \cup Y_{i+1} \cup Y_{i+2}$ 
\end{Lemma}

\begin{Lemma}\label{lem:Xi-Yi-noYi}
If $X_i \neq \phi$ then $Y_{i-1} =\phi$ or $Y_{i+2} = \phi$
\end{Lemma}

\begin{Lemma}\label{lem:3K-clique}
$Y_i$ forms a clique
\end{Lemma}

\begin{Lemma}\label{lem:si-no-share-vertex}
No two vertices from $Y_i$ can share a non-neighbor in $Y_{i-1}$ or $Y_{i+1}$
\end{Lemma}
\noindent {\it} Proof. Suppose there existed two vertices $x_1$ and $x_2$ in $Y_i$ that have a non-neighbor $y$ in $Y_{i-1}$ then there exists a $co-diamond (x_1x_2, y, 2)$.

\begin{Lemma}\label{lem:yi-adjacency-yi2}
$Y_i \;\circled{0}\; Y_{i+2} \cup Y_{i+3} $
\end{Lemma}

\begin{Lemma}\label{lem:yi-miss-two-neighbors}
If $y_i$ from $Y_i$ has a nonneighbor in $Y_{i+1}$ (in $Y_{i-1}$ respectively) then $yi \;\circled{1}\; Y_{i-1}$ (in $Y_{i+1}$ respectively).
\end{Lemma}

\begin{Lemma}\label{lem:yi-force-join}
If $Y_i \neq \phi$ then $Y_{i+2} \;\circled{1}\; Y_{i+3}$
\end{Lemma}
\noindent {\it Proof.} Let $x$ be a vertex from $Y_i$, $y_1$ be a vertex from $Y_{i+2}$, and $y_2$ be a vertex from $Y_{i+3}$. If $y_1y_2\not\in E$ then there is a $co$-$diamond (y_1, y_2, ix)$. 

\begin{Lemma}\label{lem:max-clique-yi}
If maximal clique $ = (Y_i \cup i \cup i+1)$ then every $y \in Y_i$ has a non-neighbor $\in Y_{i-1} \cup Y_{i+1}$ and $Y_{i-1} = \phi$ or $Y_{i+1} = \phi$
\end{Lemma}
\noindent {\it Proof.} Since maximal clique $ = (Y_i \cup i \cup i+1)$ then every vertex $v \in Y_{i-1} \cup Y_{i+1}$ has a non-neighbor in $Y_i$ else $v$ would be part of the maximal clique. By Lemma \ref{lem:yi-miss-two-neighbors} and Lemma \ref{lem:si-no-share-vertex} $Y_{i-1} = \phi$ or $Y_{i+1} = \phi$.

\begin{Lemma}\label{lem:add-2vertex}
If $|X_0 \cup X_1 \cup X_2 \cup X_3 \cup X_4| = 1$ then $\omega(G) =  \varphi(G)$
\end{Lemma}
\noindent {\it Proof.} By Lemma \ref{lem:Xi-Yi-noYi} at least one $Y_i = \phi$, assume $Y_3 = \phi$ and are adding a vertex $k \in X_0$. Let $Z_i$ denote the set of vertices from $Y_i$ with a non-neighbor in $Y_{i+1}$. There are five distinct cases two deal with.

(I) If the maximal clique $\subset Y_0 \cup Y_4 \cup 4 \cup 0$. Adding $k$ then result in $\omega(G)$ to increase by one since $k \;\circled{1}\; Y_0 \cup Y_4$ by Lemma \ref{lem:2k-join-3k}, $k4 \in E$ and $k0 \in E$. If $\phi(G) > \omega(G)$ before adding $k$ then $\phi(G)$ stays the same else it increase by one.

(II) If the maxmial clique $\subset Y_1 \cup 0 \cup 1$. Adding $k$ will result in $\omega(G)$ to remain the same. By Lemma \ref{lem:max-clique-yi} either $Y_0 = \phi$ or $Y_2 = \phi$. If $Y_2 = \phi$ then $Y_4 \;\circled{1} Y_0$ by Lemmas \ref{lem:max-clique-yi} and \ref{lem:yi-miss-two-neighbors}. So then $|Y_1| > |Y_0| + |Y_4|$ else it is Case I. So $Y_0$ and $Y_4$ can be colored using $|Y_1| - 1$ colors. Vertex $2$ uses color of vertex $0$ and vertex $3$ uses color of $1$. Vertex $4$ can use the remaining color from $Y_1$. Since $3k \not \in E$ and $1k \not \in E$ $k$ can uses the same color. If $Y_0 = \phi$ then $|Y_1| > |Y_0$ (else case I) and $|Y_1| \geq |Y_2|$. $Y_2$ uses $|Y_1|$ colors and $Y_0$ uses $|Y_1| - 1$ colors. Vertex $2$ uses color of vertex $0$ and vertex $3$ uses color of $1$. Vertex $4$ can use the remaining color from $Y_1$. Since $3k \not \in E$ and $1k \not \in E$ $k$ can uses the same color. So $G$ was colored using $\omega(G)$ colors.

(III) If the maximal clique $\subset Y_0 \cup Y_1 \cup 0 \cup 1$. Adding $k$ then will result in $\omega(G)$ to remain the same. Let $A$ denote the set of vertices from $Y_0 \cup Y_1 \in$ the maximal clique. $|A| > |Y_0|$ and $|A| > |Y_1|$ else it is Case I or Case II. $|Y_1 \not\in A| < |Y_0 \in A|$ $(|A| > |Y_1|)$ and can be colored using colors from $Y_0 \in A$ since every $Y_1 \not \in A$ has a non-neighbor in $Y_0$. $|Y_0 \not\in A| < |Y_1 \in A|$ $(|A| > |Y_0|)$ and can be colored using colors from $Y_1 \in A$ since every $Y_0 \not \in A$ has a non-neighbor in $Y_1$. 
$|Y_2| <= |A|$ else it would be the maximal clique.  If $|Y_2| = |A|$ then by Lemma \ref{lem:si-no-share-vertex} $|Z_1| = |Y_1|$ and $|Y_1 \not \in| = 0$. $Y_2$ can be colored using the colors from $A$. If $|Y_2| < |A|$ then $|Z_1| < |Y_1|$ but by Lemma \ref{lem:si-no-share-vertex} $|Z_1| = |Y_2|$ and $Y_2$ can be colored using the colors from $Z_1$. $|Y_4| + |Y_0| - |Z_4| < |A|$ else it is Case I. $Y_4$ can be colored using the $A$ - 1 colors from $A$. Vertex $2$ can uses color of vertex $0$, vertex $3$ can use color of vertex $1$  and vertex $4$ can use the remaining color from $A$. Since $3k \not \in E$ and $1k \not \in E$ $k$ can uses the same color. So $G$ was colored using $\omega(G)$ colors.

(IV) If the maximal clique $\subset Y_2 \cup 2 \cup 3$. Adding $k$ then will result in $\omega(G)$ to remain the same. By Lemma \ref{lem:yi-force-join} $|Z_4| = 0$. $|Y_2| > |Y_4| + |Y_0|$ and $|Y_2| > |Y_0| + |Y_1| - |Z_0|$ else it is Case I or III.  $Y_4$, and $Y_0$ can be colored using $|Y_2| - 1$ colors. By Lemma \ref{lem:max-clique-yi} every vertex $v \in Y_1$ has a non-neighbor in $Y_2$ and using $|Y_2| > |Y_0| + |Y_1| - |Z_0|$ then $Y_1$ can be colored using $|Y_2|$ colors. Vertex $0$ uses color of vertex $2$ and vertex $1$ uses color of $3$. Vertex $4$ can use the remaining color from $Y_2$. Since $3k \not \in E$ and $1k \not \in E$ $k$ can uses the same color. So $G$ was colored using $\omega(G)$ colors.

(V) If the maximal clique $\subset Y_2 \cup Y_1 \cup 2 \cup 1$. Adding $k$ then will result in $\omega(G)$ to remain the same. Let $A$ denote the set of vertices from $Y_1 \cup Y_2 \in$ the maximal clique. By Lemma \ref{lem:yi-force-join} $|Z_4| = 0$.  $|A| > |Y_4| + |Y_0|$, $|A| > |Y_0| + |Y_1| - |Z_0|$, $|A| > |Y_1|$, and $|A| > |Y_2|$ else it is Case I-IV.  $Y_4$ and $Y_0$ can be colored using $|A| - 1$ colors.
$|Y_2 \not\in A| < |Y_1 \in A|$ $(|A| > |Y_2|)$ and can be colored using colors from $Y_1 \in A$ since every $Y_2 \not \in A$ has a non-neighbor in $Y_1$ and $|A| > |Y_0| + |Y_1| - |Z_0|$ . $|Y_1 \not\in A| < |Y_2 \in A|$ $(|A| > |Y_1|)$ and can be colored using colors from $Y_2 \in A$ since every $Y_1\not \in A$ has a non-neighbor in $Y_2$. Vertex $0$ uses color of vertex $2$ and vertex $3$ uses color of $1$. Vertex $4$ can use the remaining color from $A$. Since $3k \not \in E$ and $1k \not \in E$ $k$ can uses the same color. So $G$ was colored using $\omega(G)$ colors.

Therefore can add $k$ vertex and $\omega(G) = \varphi(G)$.

\begin{Lemma}\label{lem:2-xi-2yi-join}
if $|N_i| = 1,\; |N_{i+1}| = 1,\;$ then all $Y_j \;\circled{1}\; Y_{j+1},\; j \in {i+1, i+2}$
\end{Lemma}
\noindent {\it Proof.} By Lemma \ref{lem:2k-join-3k} $N_i \;\circled{1} Y_{i+1} \cup Y_{i+2}$ and $N_{i+1} \;\circled{1} Y_{i+1} \cup Y_{i+2}$. Let $y_1$ be a vertex from $Y_{i+1}$ and $y_2$ be a vertex from $Y_{i+2}$. If $y_1y_2 \not \in E$ then a $C_4 (y_1,N_i,y_2, i+2)$ but $G$ is {\CCD}-free. Let $y_1$ be a vertex from $Y_{i+2}$ and $y_2$ be a vertex from $Y_{i+3}$. If $y_1y_2 \not \in E$ then a $C_4 (y_1,N_{i+1},y_2, i+3)$ but $G$ is {\CCD}-free.


\begin{Lemma}\label{lem:add-two-2vertex}
If $|X_0 \cup X_1 \cup X_2 \cup X_3 \cup X_4| = 2$ then $\omega(G) =  \phi(G)$
\end{Lemma}
\noindent {\it Proof.}By Lemma \ref{lem:Xi-Yi-noYi} at least $Y_i = \phi$ and $Y_{i+1} = \phi$, assume $Y_3 = \phi$ and $Y_4 = \phi$ and are adding a vertex $k_1 \in X_0$ and $k_2 \in X_1$. Let $Z_i$ denote the set of vertices from $Y_i$ with a non-neighbor in $Y_{i+1}$. There are two distinct cases two deal with.

(I) The maximal clique $\subset Y_0 \cup Y_1 \cup 0 \cup 1$. Adding $k_2$ and $k_1$ then result in $\omega(G)$ to increase by one since $k_2 \;\circled{1}\; Y_0 \cup Y_1$ by Lemma \ref{lem:2k-join-3k}, $k_20 \in E$ and $k_21 \in E$. $k_2$ can be colored with the new color. By Lemma \ref{lem:2-xi-2yi-join}  $|Y_0| + |Y_1| \geq |Y_2| +|  |Y_1|$. So $Y_2$ can be colored using the colors from $Y_0$. Vertex $2$ uses color of $0$, vertex $3$ uses color of $1$. Vertex $4$ can use color of $k_2$ since $4k_2 \not \in E$ and since $3k \not \in E$ and $1k \not \in E$ $k$ can uses the same color. So $G$ was colored using $\omega(G)$ colors.

(II) The maximal clique $\subset Y_1 \cup Y_2 \cup 1 \cup 2$. Adding $k_2$ and $k_1$ will result in $\omega(G)$ to remain the same. By Lemma \ref{lem:2k-join-3k} $k_2 \;\circled{1} Y_1 \cup Y_2$ and $k_1 \;\circled{1} Y_1$. By Lemma \ref{lem:2-xi-2yi-join} $|Y_2| + |Y_1| > |Y_1| + |Y_0|$ else it is covered in Case I. $Y_2$ be be colored with $|Y_2| - 1$ colors. Vertices $4$ and $k_2$ uses the color from $2$. vertices $3$ and $k_1$ use the color from $1$. Vertex $0$ uses the remaining color from $Y_2$. So $G$ was colored using $\omega(G)$ colors.



\begin{center}
{\bf Acknowledgement}
\end{center}
This work was done by authors  Laurier University. The authors A.M.H. and C.T.H. were each supported by individual NSERC Discovery Grants. D.J.F was supported by an NSERC Undergraduate Student Research Award.


\clearpage
\begin{thebibliography}{99}


\bibitem{BrEnLeLo}
    A.~Brandstadt, J.~Engelfriet, H.~Le, and V.~Lozin. Clique-Width for $4$-Vertex Forbidden Subgraphs.  {\sl SIAM
     Journal on Discrete Mathematics} 26 (2006) 1682--1708.

\end{thebibliography}

\end{document}
