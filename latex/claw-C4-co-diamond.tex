\documentclass[12pt]{article}
\usepackage{latexsym}
\usepackage{tikz}
\usepackage{tkz-graph}
\usepackage{tikz}
\newcommand*\circled[1]{\tikz[baseline=(char.base)]{
            \node[shape=circle,draw,inner sep=2pt] (char) {#1};}}
\usetikzlibrary{shapes}

\parskip=3pt

\setlength{\textheight}{8.5in}
\setlength{\textwidth}{6in}
\setlength{\topmargin}{0in}
\setlength{\oddsidemargin}{0in}
\setlength{\evensidemargin}{0in}

\newtheorem{Theorem}{Theorem}[section]
\newtheorem{Corollary}[Theorem]{Corollary}
\newtheorem{Lemma}[Theorem]{Lemma}
\newtheorem{Observation}[Theorem]{Observation}

\def\lc{\left\lceil}   
\def\rc{\right\rceil}
\def\inst#1{$^{#1}$}
\def\CCD{($claw$, $C_4$, $co$-$diamond$)}

\title{On k-critical {\CCD}-free graphs}

\author{
	Dallas J. Fraser\inst{1}
	\and Ang\`ele M. Hamel'\inst{1}
	\and Ch\'inh T. Ho\`ang\inst{1}
}
\begin{document}
\maketitle

\begin{center}
{\footnotesize

\inst{1}, Department of Physics and Computer Science, Wilfrid Laurier
University, \\Waterloo, Ontario, Canada}

\end{center}

\begin{abstract}
The coloring of {\CCD}-free graph is polynomial.

\noindent{\em Keywords}: Graph coloring, $claw$, $C_4$, $co$-$diamond$
\end{abstract}


\section{Introduction}\label{sec:intro}

\begin{Lemma}\label{lem:join-critical}
Let $G=(V,E)$ be any graph.  Suppose that $V$ admits a partition into
two non-empty sets $V_1$ and $V_2$ such that $V_1$ is complete to
$V_2$.  Then $G$ is critical if and only if the two graphs $G[V_1]$
and $G[V_2]$ are critical.
\end{Lemma}

\section{Oberservations}\label{sec:observations}
In this section, we conclude some observations used to prove that there polytime algorithm for {\CCD}-free graphs. For the following let $G$ be a {\CCD}-free graph.

\begin{Lemma}\label{lem:cok4-reduce-codiamond}
If $G$ contains a $\overline{K_4}$ then $G$ is a $\overline{K_4}$.
\end{Lemma}
\noindent {\it Proof}. Let $G$ contain a $\overline{K_4}$ with vertices 1, 2, 3, 4. Then $G$ has no $k$-vertex for $k \in {1,2,3,4}$ since $G$ is {\CCD}-free so $G$ must be a $\overline{K_4}$. 

\begin{Lemma}\label{lem:odd-hole-free}
$G$ cannot contain an induced $C_\ell$, $\ell \geq 7$.
\end{Lemma}
\noindent {\it Proof}. Let $G$ be a $C_\ell$, $\ell \geq 7$ with vertices 1, 2, ...,$\ell - 1,\; \ell$. This forms a $co$-$diamond$ with (1,2) and ($4,\; \ell - 1$) but $G$ is {\CCD}-free. $\Box$

\begin{Lemma}\label{lem:anti-odd-hole-free}
$G$ cannot contained an induced $\overline{C_\ell}$, $\ell \geq 7$
\end{Lemma}
\noindent {\it Proof}. Let $G$ be a $\overline{C_\ell}$, $\ell \geq 7$ with vertices 1, 2, ...,$\ell_{-1},\; \ell$. This forms a $C_4$ with (1,2 $\ell-2,\; \ell-1$).

\begin{Lemma}\label{lem:c5-k-vertex}
$G$ which contains a $C_5$ has no $k$-vertex for $k \in {0, 1, 2, 4}$
\end{Lemma}
\noindent {\it Proof}. Let $G$ be a $C_5$ with vertices 1, 2, ..., 5. $G$ has no $0$-vertex and $1$-vertex since is it $co$-$diamond$-free. $G$ has no $4$-vertex since it is $C_4$-free.

For the following let $G$ contain a $C_5$ with vertices 1, 2, ..., 5. Let $X_i$ denote the set of $2$-vertices for $C$ adjacent to $i, i+1$, $Y_i$ denote the the set of $3$-vertices for $C_5$ adjacent to $i-1, i, i+1$ and $U$ denote the set of $5$-vertices for $C_5$.

\begin{Lemma}\label{lem:max-2-xi}
$|X_i| \leq 1$
\end{Lemma}
\noindent {\it Proof}. Let $x$ and $y$ be vertices from $X_i$ such that $x \neq y$. If $xy \in E$ there is a $co-diamond (x, y, i-1, i+3)$ and if $xy \not \in E$ then there is a $claw (x, y, i, i-1)$, so $x = y$ and $|X_i| = 1$ since $G$ is {\CCD}-free.

\begin{Lemma}\label{lem:xi-no-xi2}
If $X_i \neq \phi$ then $X_{i+2} \cup X_{i+3} = \phi$
\end{Lemma}
\noindent {\it Proof}. Let $x$ be a vertex from $X_i$ and $y$ be a vertex from $X_{i+2}$. If $xy \in E$ there is a $C_4 (x, y, i+1, i+2)$ and if $xy \not \in E$ there is a $co-diamond (x,i+1, y, i-1)$, so $X_{i+2} = \phi$ since $G$ is {\CCD}-free. Let $x$ be a vertex from $X_i$ and $y$ be a vertex from $X_{i+3}$. If $xy \in E$ there is a $C_4 (x, y, i, i+4)$ and if $xy \not in E$ there is a $co-diamond (x, i, y, i+2)$, so so $X_{i+3} = \phi$  since $G$ is {\CCD}-free. 

\begin{Lemma}\label{lem:max-2-2K}
$|X_1 \cup X_2 \cup X_3 \cup X_4 \cup X_5| \leq 2$
\end{Lemma}
\noindent {\it Proof.} By Lemmas \ref{lem:max-2-xi} and \ref{lem:xi-no-xi2} there can be a most two $2$-vertices on $C$.

\begin{Lemma}\label{lem:2k-join-3k}
$X_i \;\circled{1}\; Y_{i-1} \cup Y_{i} \cup Y_{i+1} \cup Y_{i+2}$ 
\end{Lemma}
\noindent{\it Proof.} Let $x$ be a vertex from $X_i$ and $y$ be a vertex from $Y_{i-1} \cup Y_{i} \cup Y_{i+1} \cup Y_{i+2}$. There is always an edge between $y$ and $C_j$ such that $xC_j \not\in E$ and there is a vertex $z$ such that $xz \not \in E$ and $yz \not \in E$. If $xy \not \in E$ then a $co$-$diamond (yC_j, x, z)$ so $xy \in E$.

\begin{Lemma}\label{lem:Xi-Yi-noYi}
If $X_i \neq \phi$ then $Y_{i-1} =\phi$ or $Y_{i+2} = \phi$
\end{Lemma}
\noindent{\it Proof.} Let $x$ be a vertex from $X_i$, $y_1$ be a vertex from $Y_{i-1}$, $y_2$ be a vertex from $Y_{i+2}$. By Lemma \ref{lem:2k-join-3k} $xy_1 \in E$ and $xy_2 \in E$ and by Lemma \ref{lem:yi-adjacency-yi1} $y_1y_2 \not \in E$ but then there is a $C_4 (x, y_1, y_2, i+3)$. So $Y_{i-1} = \phi$ or $Y_{i+2} = \phi$.

\begin{Lemma}\label{lem:3K-clique}
$Y_i$ forms a clique
\end{Lemma}
\noindent {\it Proof}. Let $x$ and $y$ be vertices from $Y_i$ such that $x \neq y$. If $xy \not \in E$ there is a $claw (x, y, i-1, i)$ so $xy \in E$ since $G$ is {\CCD}-free.

\begin{Lemma}\label{lem:yi-adjacency-yi2}
$Y_i \;\circled{0}\; Y_{i+2} \cup Y_{i+3} $
\end{Lemma}
\noindent {\it Proof}. Let $x$ be a vertex from $Y_i$ and $y$ be a vertex from $Y_{i+2}$. If $xy \in E$ there is a $C_4 (i-1, x, y, i+3)$, so $xy \not \in E$ since $G$ is {\CCD}-free. Let $x$ be a vertex from $Y_i$ and $y$ be a vertex from $Y_{i+3}$. If $xy \in E$ there is a $C_4 (i+1, x, y, i+2)$, so $xy \not \in E$ since $G$ is {\CCD}-free.

\begin{Lemma}\label{lem:yi-miss-two-neighbors}
If $y_i$ from $Y_i$ has a nonneighbor in $Y_{i+1}$ (in $Y_{i-1}$ respectively) then $yi \;\circled{1}\; Y_{i-1}$ (in $Y_{i+1}$ respectively).
\end{Lemma}
\noindent {\it} Proof. By Lemma \ref{lem:cok4-reduce-codiamond} $G$ can only contain $\overline{K_4}$ if it is a $\overline{K_4}$. The proof from \cite{BrEnLeLo} holds.

\begin{Lemma}\label{lem:yi-force-join}
If $Y_i \neq \phi$ then $Y_{i+2} \;\circled{1}\; Y_{i+3}$
\end{Lemma}
\noindent {\it Proof.} Let $x$ be a vertex from $Y_i$, $y_1$ be a vertex from $Y_{i+2}$, and $y_2$ be a vertex from $Y_{i+3}$. If $y_1y_2\not\in E$ then there is a $co$-$diamond (y_1, y_2, ix)$. 

Let $B_i$ be a bag of a buoy and $S_i$ be the set of vertices such that $S_iB_{i-1} \not \in E$ or $S_iB_{i+1} \not \in E$


\begin{Lemma}\label{lem:si-no-share-vertex}
No two vertices from $S_i$ can share a non-neighbor in $B_{i-1}$ or $B_{i+1}$
\end{Lemma}
\noindent {\it} Proof. Suppose there existed two vertices $x_1$ and $x_2$ that have a non-neighbor $y$ in $B_{i-1}$ then there exists a $co-diamond (x_1x_2, y, B_2)$.

\begin{Lemma}\label{lem:3-si-limit}
If $x$ from $S_i$ has a non-neighbor in $B_{i-1}$ then $|B_{i+2}| = 1$ and $S_{i+2} = \phi$.  
\end{Lemma}
\noindent {\it} Proof. Let $x$ be a vertex from $S_i$ which has a non-neighbor in $B_{i-1}$. If $|B_{i+2}| > 1$ or  $|Y_{i+2}| > 0$  then there would exists a $co$-$diamond$.

\begin{Lemma}\label{lem:si-limit}
The set $X$ of $S_i$ with a non-neighbor in $B_{i+1}$ is $|X| \leq |B_{i+1}|$ 
\end{Lemma}
\noindent {\it} Proof. Let $X$ be the set of vertices of $S_i$ with a non-neighbor in $B_{i+1}$. If $|X| > |B_{i+1}|$ then there would exist a vertex $y$ such that two vertices $x_1$ and $x_2$ from $X$ which are non-adjacent to $y$. By Lemma \ref{lem:3K-clique} $X$ forms a clique and hence a $co-diamond (y, i+3, x_1x_2)$

\begin{Lemma}\label{lem:max-3-si}
There can be a most three sets of $S_i \neq \phi$. 
\end{Lemma}
\noindent {\it} Proof. By Lemma \ref{lem:3-si-limit} if there two $S_i \neq \phi$ and $S_j \neq \phi$ then  $S_{i+3} = \phi$ and $S_{j+3} = \phi$. Furthermore $S_i \neq \phi$ where $i \in \{i, i+1, i+2\}$. 

From now on assume $S_3 \cup S_4 = \phi$ and $|B_3| = 1$ and $|B_4| = 1$. Let $A$ denote the set of vertices from $S_1$ with a non-neighbor in $B_2$ and $C$ denote the set of vertices from $S_1$ with a non-neighbor in $B_0$. Note that $S_0$ and $S_2$ have a non-neighbor in $B_1$.

\begin{Lemma}\label{lem:max-s0-s2}
$|B_1| > |S_0| + |S_2|$
\end{Lemma}
\noindent {\it} Proof. Suppose $|B_1| < |S_0| + |S_2|$ then by Lemma \ref{lem:si-no-share-vertex} there exists three vertices $x \in B_1, y_1 \in S_0,$ and $y_2 \in S_2$ such that $xy_1 \not \in E$ and $xy_2 \not \in E$. Then there is a $co-diamond (x, y_1, y_2B_3)$. Suppose $|B_1| = |S_0| + |S_2|$ then there exists a vertex $v \in S_0 \cup S_2$ such that $v1 \not \in E$ and therefore $v \in N_0 \cup N_2$ and $v \not \in S_0 \cup S_2$.

\begin{Lemma}\label{lem:omega-equals-chromatic}
If $G$ contains $S_i \neq \phi$ then $\omega(G) = \chi(G)$
\end{Lemma}
\noindent {\it} Proof. By Lemma \ref{lem:si-limit} $|A| \leq |B_2$| and $|C| \leq |B_0|$. If $|A| + |B_0| \geq |C| + |B_2|$ then the $\omega(G) = B_1 \cup B_0 \cup A$ else $\omega(G) = B_1 \cup B_2 \cup C$. Without loss of generality assume $|A| +|B_0| \geq |C| + |B_2|$. By Lemma \ref{lem:max-s0-s2} every vertex from $S_0$ and $S_2$ has a non-neighbor in $B_1$ and by Lemma \ref{lem:yi-adjacency-yi2} $S_0 \;\circled{0}\; S_2$ so $S_0 \cup S_2$ can be colored with the $|B_1| - 1$colors. Since $|C| \leq |B_0|$ and by Lemma \ref{lem:si-limit} every vertex in $C$ has a non-neighbor $\in B_0$ it can be colored with the colors from $B_0$. $B_2 \;\circled{0}\;B_0 \cup A$ and $|B_2| \leq |B_0| +|
A| - |C|$ so can color $B_2$ with colors from $B_0 \not \in C \cup A$. $B_3$ can use the remaining color from $B_1$ since $B_3 \;\circled{0}\; B_1$. $B_4$ can use a remaining color from $B_1$ since $B_4 \;\circled{0}\; B_1$ and color of $B_3 \neq$ color of $B_1$ since $S_0$ can't share a non-neighbor with $S_2$. Therefore all vertices all colored and $\omega(G) = \chi(G)$.

\begin{center}
{\bf Acknowledgement}
\end{center}
This work was done by authors  Laurier University. The authors A.M.H. and C.T.H. were each supported by individual NSERC Discovery Grants. D.J.F was supported by an NSERC Undergraduate Student Research Award.


\clearpage
\begin{thebibliography}{99}


\bibitem{BrEnLeLo}
    A.~Brandstadt, J.~Engelfriet, H.~Le, and V.~Lozin. Clique-Width for $4$-Vertex Forbidden Subgraphs.  {\sl SIAM
     Journal on Discrete Mathematics} 26 (2006) 1682--1708.

\end{thebibliography}

\end{document}
