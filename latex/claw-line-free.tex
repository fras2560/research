\documentclass[12pt]{article}
\usepackage{latexsym}
\usepackage{tikz}
\usepackage{tkz-graph}
\newcommand*\circled[1]{\tikz[baseline=(char.base)]{
            \node[shape=circle,draw,inner sep=2pt] (char) {#1};}}
\usetikzlibrary{shapes}

\parskip=3pt

\setlength{\textheight}{8.5in}
\setlength{\textwidth}{6in}
\setlength{\topmargin}{0in}
\setlength{\oddsidemargin}{0in}
\setlength{\evensidemargin}{0in}

\newtheorem{Theorem}{Theorem}[section]
\newtheorem{Corollary}[Theorem]{Corollary}
\newtheorem{Lemma}[Theorem]{Lemma}
\newtheorem{Observation}[Theorem]{Observation}

\def\lc{\left\lceil}   
\def\rc{\right\rceil}
\def\inst#1{$^{#1}$}
\def\FAM{($\overline{K_4}$, $line$)}

\title{On k-critical ({\FAM})-free graphs}

\author{
	Dallas J. Fraser\inst{1}
	\and Ang\`ele M. Hamel'\inst{1}
	\and Ch\'inh T. Ho\`ang\inst{1}
}
\begin{document}
\maketitle

\begin{center}
{\footnotesize

\inst{1}, Department of Physics and Computer Science, Wilfrid Laurier
University, \\Waterloo, Ontario, Canada}

\end{center}

\begin{abstract}
The coloring of {\FAM}-free graph is in polynomial.

\noindent{\em Keywords}: Graph coloring, $claw$, $K_5$ - $e$, $\overline{K_4}$, $line$-$graph$ 
\end{abstract}


\section{Introduction}\label{sec:intro}
Deterimining the chromatic number of a graph is a NP-hard problem. For some graphs families the problem can be solved in polynomial time. This paper looks at $\overline{K_4}$ and $line$-free graphs. For a graph to be $line$-free is has no induced subgraph of the following graphs: $claw$, $K_5$ - $e$, $\overline{Twin-C5}$, $\overline{Twin-House}$, etc..

\begin{Lemma}\label{lem:cliqueCutset}
A critical graph that is not a clique does not contain a clique
cutset.
\end{Lemma}

\section{Oberservations}\label{sec:observations}
In this section, we conclude some observations used to prove that there polytime algorithm for {\FAM}-free graphs. For the following let $G$ be a graph that is {\FAM}-free.

\begin{Lemma}\label{lem:cl-free}
$G$ contains no $C_\ell,\; \ell \geq 8$
\end{Lemma}
\noindent {\it Proof.} Since $G$ is $\overline{K_4}$-free it has no $C_\ell,\; \ell \geq 8$

\begin{Lemma}\label{lem:c7-bounded}
If $G$ contains a $C_7$ then $|G| \leq 14$
\end{Lemma}
\noindent {\it Proof.} Let $G$ contain a $C_7$. Then $G$ has no $k$-vertex for $k \in {0, 1, 2}$ since $G$ is $\overline{K_4}$-free, has no $k$-vertex for $k \in {5, 6,7}$ since $G$ is $line$-free ($claw$), and $G$ has no $k$-vertex for $k \in {3}$ since $G$ is $line$-free ($\overline{Twin-House}$). $G$ only has a $4$-vertex denoted by $Y_i$ on vertices $i,i+1,i+2,i+4$. Let $y_1$ and $y_2$ be vertices $\in Y_i,\; y_1 \neq y_2$. If $y_1y_2 \not \in E$ then there is a $claw (i+4i+5,i+4y_1,i+4y_2)$ and if $y_1y_2 \in E$ then there is a $\overline{Twin-House}(i,i+1,i+2,y_1,y_2)$ so $|Y_i| \leq 1$. Since $|Y_i| \leq 1$ $G$ can contain at most $14$ vertices.

\begin{Lemma}\label{lem:c5-kvertex}
If $G$ contains $C_5$ then $G$ has no $k$-vertex $\in {1,3,5}$
\end{Lemma}
\noindent {\it Proof.} Let $G$ contain a $C_5$. Then $G$ has no $k$-vertex for $k = 1$ since $G$ is $line$-free ($claw$), $k = 3$ since $G$ is $line$-free ($\overline{Twin-C5}$), $k = 5$ since $G$ is $line$-free ($W_5$). 

For the following let $G$ contain a $C_5$. Let the $0$-vertex set be denoted by $R$, let the $2$-vertex set on vertices $i, i+1$ be denoted by $X_i$, and let the $4$-vertex set on the vertices $i, i+1, i+2, i+3$ be denoted by $Y_i$. Let $X$ denote the set of $2$-vertex and $Y$ denote the set of $4$-vertex.

\begin{Lemma}\label{lem:c5-4vertex-bounded}
$|Y_i| \leq 1$
\end{Lemma}
\noindent {\it Proof.} Let $y_1$ and $y_2$ be vertices $\in Y_i,\; y_1 \neq y_2$. If $y_1y_2 \not \in E$ then there is a $claw (ii-1, iy_1,iy_2)$. If $y_1y_2 \in E$ then there is a $\overline{Twin-House} (i,i+1,i+2,y_1,y_2)$. So $|Y_i| \leq 5$.

\begin{Lemma}\label{lem:yi-cojoin-yi1}
$Y_i \;\circled{0}\; Y_{i+1}$
\end{Lemma}
\noindent {\it Proof.} Let $y_1$ be a vertex from $Y_i$ and $y_2$ be a vertex from $Y_{i+1}$. If $y_1y_2 \in E$ then there is a $K_5 - e (i+1, i+2, i+3, y_1, y_2)$.

\begin{Lemma}\label{lem:yi-join-yi2}}
$Y_i \;\circled{1}\; Y_{i+2}$
\end{Lemma}
\noindent {\it Proof.} Let $y_1$ be a vertex from $Y_i$ and $y_2$ be a vertex from $Y_{i+2}$. If $y_1y_2 \in E$ then there is $\overline{A} (i+1, i+2, i+3, i+4, y_1, y_2)$.

\begin{Lemma}\label{lem:xi-clique}
$X_i$ forms a clique
\end{Lemma}
\noindent {\it Proof.} Let $x_1$ and $x_2$ be vertices $\in X_i,\; x_1 \neq x_2$. If $x_1x_2 \not \in E$ then there is a $claw (ii-1, ix_1, ix_2)$. So $x_1x_2 \in E$ and $X_i$ forms a clique.



\begin{Lemma}\label{lem:r-clique}
$R$ forms a clique
\end{Lemma}
\noindent {\it Proof.} Let $r_1$ and $r_2$ be vertices from $R$. If $r_1r_2 \not \in E$ then there is a $\overline{K_4} (0, 2, r_1, r_1)$.

\begin{Lemma}\label{lem:r-join-xi}
$R \; \circled{1} \; X_i$
\end{Lemma}
\noindent {\it Proof.} Let $x$ be a vertex $\in X_i$ and $r$ be a vertex $\in R$. If $rx \not \in E$ then there is a $\overline{K_4} (r, x, i-1, i+2)$.

\begin{Lemma}\label{lem:r-limits-xi}
If $R \neq \phi$ then $|X_i| \leq 2$
\end{Lemma}
\noindent {\it Proof.} Suppose $|X_i| \geq 3$ and $R \neq \phi$. Let $x_1,\; x_2,\;$ and $x_3$ be vertices from $X_i,\; x_1 \neq x_2,\; x_1 \neq x_3,\; x_2 \neq x_3$. By Lemma \ref{lem:xi-clique} $x_1,\; x_2,\;$ and $x_3$ forms a clique. By Lemma \ref{lem:r-join-xi} there is a $K_5$ - $e$ but $G$ is $line$-free.

\begin{Lemma}\label{lem:2xi-cojoins-xj}
A vertex $\in X_i$ cannot have two neighbors in $X_j,\; i \neq ij$.
If $|X_i| \geq 2$ then $x_j$ cannot $ join \; X_i$
\end{Lemma}
\noindent {\it Proof.} Suppose $|X_j| \geq 2$ with vertices $x_1$ and $x_2$. Let $x_i$ be a vertex from $X_{i}$. If $x_ix_1 \in E$ and $x_ix_2 \in E$ then there is a $\overline{Twin-House}$.

\begin{Lemma}\label{lem:xi-limits-r}
If $X \neq \phi$ then $|R| \leq 2$ or $R \in$ a clique cutset 
\end{Lemma}
\noindent {\it Proof.} If $X_i \neq \phi,\;$ then by Lemma \ref{lem:r-join-xi} $R$ joins $X_i$ and is a cutset. If $X_i \neq \phi,\; X_j \neq \phi,\; i \neq j$. If $|R| \geq 3$ then by Lemma \ref{lem:r-join-xi} and $X_i \; \circled{1} \; X_j$ else there is a $K_5 - e$. Then by Lemma \ref{lem:2xi-cojoins-xj} $|X_i| = 1$ and $|X_j| = 1$. Then $R$ is part of a clique cutset.

\begin{Lemma}\label{lem:r-bounded}
If $R \neq \phi$ then $ |v| \leq 17$ or $G$ has a clique cutset
\end{Lemma}
\noindent {\it Proof.} If $|R| \geq 3$ then by Lemma \ref{lem:xi-limits-r} is a clique cutset. If $|R| \leq 2$ then by Lemmas \ref{lem:r-limits-xi} and \ref{lem:2xi-cojoins-xj} $|X_i| \leq,\; i \in {0,1,2,3,4}$. By Lemma \ref{lem:one-4vertex} the number of $4$-vertex$ \leq 1$. Then $|V| \leq 2 + 5 + 5 + 5$ or $|R| \geq 3$ and $G$ contains a clique cutset.

\begin{Lemma}\label{lem:r-polynomial}
If $R \neq \phi$ then $G$ can be colored in polynomial time
\end{Lemma}
\noindent {\it Proof.} If $|R| \leq 2$ then by Lemma \ref{lem:r-bounded} is bounded and if $|R| \geq 3$ then by Lemma \ref{lem:cliqueCutset} $G$ cannot be critical and can be colored in polynomial time.

For the following assume $R = \phi$. 

\begin{Lemma}\label{lem:xi-neighbor-adjacent}
$X_i$ cannot have two neighbors $\not \in X_i$ who are non-adjacent
\end{Lemma}
\noindent {\it Proof.} Let $x_i$ denote a vertex $\in X_i$. Let $x_1$ and $x_2$ neighbors of $x_i$. By Lemma \ref{lem:xi-clique} $x_1 \in X_j,\; x_2 \in X_k,\; j \neq k$. If $j=i-1$ and $k=i+1$ then there is a $\overline{A}$ if $x_1x_2 \not in E$. If $j \neq i-1$ and $k \neq j$ then there is a $claw (ix_i, x_1x_i,x_2x_i)$.

\begin{Lemma}\label{lem:one-clique}
If only one $X_i \neq \phi,\; i \in {0, 1, 2, 3, 4}$ then $\omega(G) = \varphi(G)$.
\end{Lemma}
\noindent {\it Proof.} $X_i$, $i$ and $i+1$ forms a clique. Vertex $i+2$ can share a color with a vertex from $X_i$, vertex $i+3$ can share a color with $i$ and vertex $i+4$ can a color with $i+1$.

\begin{Lemma}\label{lem:xi-xi2-clique}
If only $X_i \neq \phi,\; X_{i+1} \neq \phi,\; i \in {0, 1, 2, 3, 4}$ then $\omega(G) = \varphi(G)$.
\end{Lemma}
\noindent {\it Proof.} $X_i$, $i$, and $i+1$ forms a clique.$|X_i| \geq |X_{i+1}$ then $X_{i+1}$ can be colored with vertices from $X_i$ since by Lemma \ref{lem:2xi-cojoins-xj} each vertex $\in X_{i+1}$ can only be adjacent to one vertex in $\in X_{i}$. Vertex $i+2$ can be colored with same color as vertex $i$, vertex $i+3$ can be colored with the same color as vertex $i+1$, and vertex $i+4$ can be colored with a vertex from $X_i$.

For the following assume there are three $X_i \neq \phi$ for three distinct $i$.

\begin{Lemma}\label{lem:xi-xi2-xi3-clique}
If $|X_i| \geq 3$ then $|X_j| = 1$ and 
\end{Lemma}
\noindent {\it Proof.}

\begin{center}
{\bf Acknowledgement}
\end{center}
This work was done by authors  Laurier University. The authors A.M.H. and C.T.H. were each supported by individual NSERC Discovery Grants. D.J.F was supported by an NSERC Undergraduate Student Research Award.


\clearpage
\begin{thebibliography}{99}

\end{thebibliography}

\end{document}
