\begin{Lemma}\label{lem:P58V}
If $G$ contains a $P5$ then $G$ has at most 8 vertices or it is disconnected.
\end{Lemma}
\noindent {\it Proof.} Expanding \cite{BraFud2002} where it was shown $P_5$ has no 1-vertex, 2-vertex, 5-vertex and possible $k$-vertices for $P$ with $k \geq 3$ are the sets $A$ = $N_1,_2,_4,_5$, $B$ = $N_1,_3,_4$, and $C$ = $N_2,_3,_5$. Moreover since $G$ is $claw$-free, there are no edges between $A$ and $B$ ($AB$, $A$2, $A$5), between $A$ and $C$ ($AC$, $A$1, $A$4), and between $B$ and $C$ ($BC$, $B$1, $B4$). The sets can have at most one vertex. Let $A$ contains two vertices $x$ and $y$. If $xy \in E$ then $co$-$claw$ ($xy$, $y1$, $1x$, $3$) but if $xy \not\in E$ then $claw$ ($1x$, $1y$, $16$)$-$contradiction. The same argument applies to $B$ and $C$. 3,4-vertex and 0-vertex sets cannot both $!= \phi$ esle there would be a $co$-$claw$. If the set of 3,4-vertex $!= \phi$ then $G$ has at most 8 vertices. If the set of 0-vertex $!= \phi$ then $G$ is a disconnected.

From now on, assume that $G$ is $C_k$- and $\overline{C_k}$-free for $k \geq 5$

\begin{Lemma}\label{lem:P6Path}
If $G$ contains a $P_l,$ $l \geq 6$, then $G$ is such a path or is disconnected
\end{Lemma}
{\it Proof.} Expanding \cite{BraFud2002} Theorem 2 Claim 4 where it was shown $P_l$, $l \geq 6$ had no $k$-vertex for $k \geq 1$. If the set of 0-vertex $= \phi$ then $G$ is disconnected else it is a $P_l$.  $\Box$

From now on, assume that $G$ is $C_l$-free and $\overline{C_l}$-free $l \geq 5$, as well as $P_l$- and $\overline{P_l}$-free, $l \geq 6$.