\documentclass[12pt]{article}
\usepackage{latexsym}
\usepackage{tikz}
\usepackage{tkz-graph}
\usepackage{tikz}
\newcommand*\circled[1]{\tikz[baseline=(char.base)]{
            \node[shape=circle,draw,inner sep=2pt] (char) {#1};}}
\usetikzlibrary{shapes}

\parskip=3pt

\setlength{\textheight}{8.5in}
\setlength{\textwidth}{6in}
\setlength{\topmargin}{0in}
\setlength{\oddsidemargin}{0in}
\setlength{\evensidemargin}{0in}

\newtheorem{Theorem}{Theorem}[section]
\newtheorem{Corollary}[Theorem]{Corollary}
\newtheorem{Lemma}[Theorem]{Lemma}
\newtheorem{Observation}[Theorem]{Observation}

\def\lc{\left\lceil}   
\def\rc{\right\rceil}
\def\inst#1{$^{#1}$}
\def\CCD{($claw$, $co$-$diamond$)}

\title{Coloring {\CCD}-free graphs}

\author{
	Dallas J. Fraser\inst{1}
	\and Ang\`ele M. Hamel'\inst{1}
	\and Ch\'inh T. Ho\`ang\inst{1}
}
\begin{document}
\maketitle

\begin{center}
{\footnotesize

\inst{1}, Department of Physics and Computer Science, Wilfrid Laurier
University, \\Waterloo, Ontario, Canada}

\end{center}

\begin{abstract}
Coloring of {\CCD}-free graphs
\noindent{\em Keywords}: Graph coloring, $claw$, $co$-$diamond$
\end{abstract}

\section{Oberservations}\label{sec:observations}
In this section, we conclude some observations about the structure of a {\CCD}-free graph. Let $G$ be a {\CCD}-free graph. Let $x$ denote a vertex from $G$, let $N_x$ denote the neighbors of $x$ and $M_x$ denote the non-neighbors of $x$.
A cross join is a join of two equal size sets $S_1 \cup S_2$ such that every vertex in $S_1$ has a neighbor in $S_2$ and every vertex in $S_2$ has a neighbor in $S_1$.

\begin{Observation}\label{obs:mx-k-partite}
$M_x$ forms a $k$-partite graph
\end{Observation}
For the following observations let $M_j$ denote the set of vertices from $M_x$ that forms a stable set. Let $J$ be the number of $M_j$ sets.

\begin{Observation}\label{obs:alpha-nx}
$\alpha(N_x) \leq 2$
\end{Observation}
\noindent {\it Proof.} Suppose $\alpha(N_x) > 2$ formed by vertices $n_1,n_2, ..$ then there is a $claw (xn_1, xn_2, xn_3)$.

\begin{Observation}\label{obs:nx-mj}
A vertex from $N_x$ can have at most one non-neighbor in $M_j$
\end{Observation}
\noindent {\it Proof.} Suppose $n$ from $N_x$ had two non-neighbors $m_1$ and $m_2$ from $M_j$. Then there is a $co$-$diamond (nx, m_1, m_2)$.

For the following let $N_i$ denote the pairs of vertices from $N_x$ that are non-adjacent. Let $I$ be the number of $N_i$ sets. Note the $|N_i| = 1$ or $|N_i| = 2$

\begin{Observation}\label{obs:ni-join-1}
If $|N_i| = 1$ then $N_i \;\circled{1}\; N_j$ such that $|N_j| = 1$
\end{Observation}
\noindent {\it Proof.} If $|N_i| = 1$ and $|N_j| = 1$ then if $N_iN_j \not \in E$ then $N_i \cup N_j$ forms a $N_i$ such that $|N_i| = 2$.

\begin{Observation}\label{obs:ni-1-neighbor}
If $|N_i| = 1$ then $N_i$ has one neighbor in $N_j$
\end{Observation}
\noindent {\it Proof.} Let $v$ be the vertex from $N_i$. If $v \;\circled{0}\; N_j$ then by Observation \ref{obs:ni-join-1} $|N_j| != 1$. If $|N_j| = 2$ then there is a $claw (vx, xN_j, xN_j)$ so $v$ must have a neighbor in $N_j$.

\begin{Observation}\label{obs:ni-2-neighbor}
$N_i$ at least cross joins $N_j$
\end{Observation}
\noindent {\it Proof.} If $|N_i| = 1$ and $|N_j| = 1$ then $N_iN_j \in E$ by Observation \ref{obs:ni-join-1}. If $|N_i| = 2$ and $|N_j| = 2$ then by Observation \ref{obs:ni-1-neighbor} every vertex $v_1 \in N_i$ has a neighbor in $N_j$ and every vertex $v_2 \in N_j$ has a neighbor in $N_i$.

\begin{Observation}\label{obs:nx-clique}
$\omega((N_x) = I$
\end{Observation} 
\noindent {\it Proof.} By Observations \ref{obs:ni-join-1}, \ref{obs:ni-1-neighbor}, and \ref{obs:ni-2-neighbor} $\phi(N_x) \geq I$. Since each $N_i$ is a stable set $\phi(N_x) \leq I$. So by Squeeze Theroem $\omega(N_x) = I$.



\begin{Observation}\label{obs:mj-2}
$|M_j| \leq 2$
\end{Observation}
\noindent {\it Proof.} Suppose $|M_j| > 2$ then by Observation \ref{obs:nx-mj} a vertex $n$ from $N_x$ will be adjacent to at least two vertices $m_1$ and $m_2$ from $M_j$. Then there is a $claw (nx, nm_1, nm_2)$.

\begin{Observation}\label{obs:alpha-mx}
$\alpha(M_x) \leq 2$
\end{Observation}
\noindent {\it Proof.} By Observations \ref{obs:mx-k-partite} and \ref{obs:mj-2} the $\alpha(M_x) \leq 2$.

For following let $n_1$ and $n_2$ denote two vertices from $N_i$. Let $m_1$ and $m_2$ denote two vertices from $M_j$.
\begin{Observation}\label{obs:ni-cant-join-mi}
$n_1$ can't $\;\circled{1} m_1 \cup m_2$ 
\end{Observation}
\noindent {\it Proof.} Suppose $n_1m_2 \in E$ and $n_1m_2 \in E$ then there is a $claw(xn_1, n_1m_1, n_1m_2)$

\begin{Observation}\label{obs:mx-disjoin}
$|M_x \;\circled{0} N_x| \leq 1$
\end{Observation}
\noindent {\it Proof.} Suppose $|M_x \;\circled{0} N_x| > 1$. Then if $|N_x| >= 1$ there is a $co$-$diamond (nx, m_1, m_2)$. If $|N_x| = 0$ then $G$ is disconnected.



\begin{center}
{\bf Acknowledgement}
\end{center}
This work was done by authors  Laurier University. The authors A.M.H. and C.T.H. were each supported by individual NSERC Discovery Grants. D.J.F was supported by an NSERC Undergraduate Student Research Award.


\clearpage
\begin{thebibliography}{99}


\bibitem{BrEnLeLo}
    A.~Brandstadt, J.~Engelfriet, H.~Le, and V.~Lozin. Clique-Width for $4$-Vertex Forbidden Subgraphs.  {\sl SIAM
     Journal on Discrete Mathematics} 26 (2006) 1682--1708.

\end{thebibliography}

\end{document}
