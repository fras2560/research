\documentclass[12pt]{article}
\usepackage{latexsym}
\usepackage{tikz}
\usepackage{tkz-graph}
\usetikzlibrary{shapes}

\parskip=3pt

\setlength{\textheight}{8.5in}
\setlength{\textwidth}{6in}
\setlength{\topmargin}{0in}
\setlength{\oddsidemargin}{0in}
\setlength{\evensidemargin}{0in}

\newtheorem{Theorem}{Theorem}[section]
\newtheorem{Corollary}[Theorem]{Corollary}
\newtheorem{Lemma}[Theorem]{Lemma}
\newtheorem{Observation}[Theorem]{Observation}

\def\inst#1{$^{#1}$}
\def\CCO{($claw$, $co$-$claw$)}

\title{On k-critical ($claw$, $co$-$claw$)-free graphs}

\author{
  Dallas Fraser\inst{1}
  \and Ang\`ele M. Hamel\inst{1}
  \and Ch\'inh T. Ho\`ang\inst{1}
}
\begin{document}
\maketitle

\begin{center}
{\footnotesize

\inst{1} Department of Physics and Computer Science, Wilfrid Laurier
University, \\ Waterloo, Ontario, Canada}
%\\ \texttt{choang@wlu.ca}


\end{center}
%
\begin{abstract}
A graph is $k$-critical if it is $k$-chromatic but each of its proper
induced subgraphs is ($k-1$)-colorable. There is an finite number of k-critical {\CCO} when $k = 3$. We show that the set of $k$-critical {\CCO} for some $k \geq 3$ is  finite. Our result implies the existence of a certifying alogirthm for $k$-coloring {\CCO}-free graphs. 

\noindent{\em Keywords}: Graph coloring, $claw$-free,$co$-$claw$-free graphs
\end{abstract}


\section{Introduction}\label{sec:intro}

Graph coloring is a well-studied problem in computer science and
discrete mathematics.  Determining the chromatic number of a graph
is a NP-hard problem.  But for many classes of graphs, such as
perfect graphs, the problem can be solved in polynomial time.
In 2002 \cite{BraFud2002}, research was done on  prime {\CCO}-free graphs and provided a starting point for this research.

The point of view in this article is motivated by the idea of a ``certifying algorithm''.  An algorithm is {\it certifying} if it returns with each output a simple and easily verifiable certificate that the particular output is correct.  For example, a certifying algorithm for the bipartite graph recognition would return either a 2-coloring of the input graph, thus proving that it is bipartite, or an odd cycle, thus proving it is not bipartite.  A certifying algorithm for planarity would return either an embedding of the graph in a plane, or one of the two Kuratowski subgraphs proving the input graph is not planar.

A graph is $k$-critical if it is $k$-chromatic but each of its proper
induced subgraphs is $(k-1)$-colorable.
Here we prove that the number of $k$-critical {\CCO}-free graphs is infinite for every fixed $k \geq 2$, but NOT SURE ON WORDING
In section~\ref{sec:definitions}, we
give definitions and background on our problem.  In
section~\ref{sec:updates}, we expand upon previous Lemmas \cite{BraFud2002}. In section!\ref{sec:characterization} , we give the proofs or our main results.  In section~\ref{sec:5-critical}, we demostrate how  
$k$-critical {\CCO}-free graphs are constructed for $k \in {4,5, 6}$.

\section{Definitions and background}\label{sec:definitions} 
A $k$-coloring of a graph $G=(V,E)$ is a mapping $f: V \rightarrow \{1,\ldots, k\}$ such that $f(u) \not= f(v)$ whenever $uv \in E$. Given a coloring, a {\it color class} is the set of all vertices of the same color.  The chromatic number $\chi(G)$ of a graph $G$ is the smallest $k$ such that $G$ is $k$-colorable.  $G$ is $k$-chromatic if $\chi(G) = k$.  A graph $G$ is {\em $k$-critical} if it is $k$-chromatic and none of its proper induced subgraphs is $k$-chromatic (that is, all of its proper induced subgraphs are $(k-1)$-colorable).  We say that a graph is {\it critical} if it is $k$-critical for some $k$.  Let $N(v)$ be the set of neighbors of $v$. A set $X$ of vertices of a graph $G=(V,E)$ is a {\em module} if for all $v \not\in X$, $X \subseteq N(v)$ or $N(v) \cap X = \phi$.  Module $X$ is {\em trivial} if $|X| = 1$ or $X = |V|$. Unless otherwise stated, a module in this paper is non-trivial.  A vertex of $G$ is {\it universal} if it is adjacent to every other vertex of $G$. Vertices $u,v$ are comparable if $N(u) \subseteq N(v)$, or vice versa. If $X$ is a set of vertices of $G$, then $G[X]$ denotes the subgraph if $G$ induced by $X$.  A set $A$ of vertices is {\it complete} to a set $B$ of vertices if there are all edges between $A$ and $B$.  Given two graphs $G$ and $ H$, the graph $F$ is the {\it join} of $G$ and $H$ if $F$ is obtained by taking $G$ and $H$ and joining every vertex in $G$ to every vertex in $H$ by an edge.  As usual, $K_t$ denotes the clique on $t$ vertices$-$ and $C_t$ denotes the induced cycle on $t$ vertices. The complement of $G$ is denoted by $\overline{G}$. 

A $claw$ refers to a graph G with vertices with one vertex $X$ is universal to {$y1,y2,y3$} which is a stable set. A $co$-$claw$ refers to a graph $G$ which is the complement of a $claw$. For $U \subseteq V$ let $G(U)$ denote the subgrpah of $G$ induced by $U$. Throughout this paper, all subgrpahs are understood to be induced. If $H$ is a subgraph of F then a vertex $v$ not in $H$ is called a $k-vertex$ for $H$ if $v$ has exactly $k$ neighbors in $H$. For a vertex set $U$ in $H$ with $|U| = k$, let $N_U$ denote the set of $k-vertices$ for $H$ being adjacent to all vertices in $U$.  The following lemmas have been previously established.

\begin{Lemma}\label{lem:CHMW}{\rm \cite{ChvHoa1987}}
$(P_5, \overline{P}_5, C_5)$-free graphs are perfect. $\Box$
\end{Lemma}

\begin{Lemma}\label{lem:connected}
A critical graph is connected. $\Box$
\end{Lemma}

\begin{Lemma}\label{lem:join-critical}{\rm \cite{DhaHam2014}}
Let $G=(V,E)$ be any graph.  Suppose that $V$ admits a partition into
two non-empty sets $V_1$ and $V_2$ such that $V_1$ is complete to
$V_2$.  Then $G$ is critical if and only if the two graphs $G[V_1]$
and $G[V_2]$ are critical. $\Box$
\end{Lemma}

\section{The structure of $k$-critical {\CCO}-free
graphs}\label{sec:updates}
%
In this section, we prove 5 properties of {\CCO}-free graphs and we show that $k$-critical {\CCO}-free graph is infinite for $k \geq 2$
%
\subsection{Structural results}

We begin by  establishing a number of preliminary results.
\begin{Lemma}\label{lem:C7Cycle}
If $G$ is a graph which contains an induced $C_l,$ $l \geq 7$, then $G$ itself is such a cycle or is a disconnected graph.
%\label{thm:ellcritical}
\end{Lemma}
\noindent {\it Proof}.  Expanding \cite{BraFud2002} upon Theorem 2 Claim 1 where it was shown $C_l$, $l \geq 7$ has not $k$-vertex for $k \geq 1$. If the set of 0-vertex $= \phi$ then $G$ is a $C_l$ else it is a disconnect graph.  $\Box$

\medskip



\begin{Lemma}\label{lem:C69V}
If $G$ contains a $C_6$ then $G$ has at most 9 vertices or is disconnected.
\end{Lemma}
\noindent{\it Proof.} Expanding \cite{BraFud2002} upon Theorem 2 Claim 2 where it was shown $C_6$ has no $k$-vertex for $k \in {1, 2, 3, 5, 6}$ Let $A$ = $N_1,_2,_4,_5$, $A$ = $N_2,_3,_5,_6$ and $C$ = $N_3,_4,_6,_1$ bet a partition of the set of 4-vertices since G is $co$-$claw$. There can be no edge $AB$ ($AB$, $A4$, $A1$), $AC$ ($AC$, $A2$, $A5$), $BC$ ($BC$, $B2$, $B5$) since $G$ is claw-free. The sets can have at most one vertex. Let $A$ contains two vertices $x$ and $y$. If $xy \in E$ then $co$-$claw$ ($xy$, $y1$, $1x$, $3$) but if $xy \not\in E$ then $claw$ ($1x$, $1y$, $16$)$-$contradiction. The same argument applies to $B$ and $C$. There are n -vertices adjacent to a 4-vertex  since $G$ is $claw$-free. If the set of 0-vertex $= \phi$ then $G$ is a $C_6$ with at most 9 vertices else it is a disconnect graph.  $\Box$

\medskip

\begin{Lemma}\label{lem:C5Join}
If $G$ contains a $C_5$ then:
\begin{itemize}
\item[(i)]
$G$ is a $C_5$
\item[(i)]
$G$ is the join of a $C_5$ and $H$
\item[(i)]
$G$ is a disconnected graph
\end{itemize}
\end{Lemma}
\noindent{\it Proof.} Expanding \cite{BraFud2002} Theorem 2 Claim 2 where it was shown $C-5$ has no $k$-vertex for $k \in {1,2,3,4}$ and no 0-vertex adjacent to a 5-vertex. The set of 5-vertex are complete to
 which means that the set of 5-vertices is complete to $C$. The 5-vertex and 0-vertex sets cannot both $!= \phi$ else there would be $co$-$claw$ If the set of 5-vertices $!= \phi$ then $G$ is a join of $C$ and $H$. If the set of 0-vertex $!= \phi$ then  $G$ is a disconnected graph. If 5-vertex and 0-vertex sets $= \phi$ then $G$ is a c $C_5$.  $\Box$

\medskip

From now on, assume that $G$ is $C_k$- and $\overline{C_k}$-free for $k \geq 5$

\begin{Lemma}\label{lem:P6Path}
If $G$ contains a $P_l,$ $l \geq 6$, then $G$ is such a path or is disconnected
\end{Lemma}
{\it Proof.} Expanding \cite{BraFud2002} Theorem 2 Claim 4 where it was shown $P_l$, $l \geq 6$ had no $k$-vertex for $k \geq 1$. If the set of 0-vertex $= \phi$ then $G$ is disconnected else it is a $P_l$.  $\Box$

From now on, assume that $G$ is $C_l$-free and $\overline{C_l}$-free $l \geq 5$, as well as $P_l$- and $\overline{P_l}$-free, $l \geq 6$.

\begin{Lemma}\label{lem:P58V}
If $G$ contains a $P5$ then $G$ has at most 8 vertices or it is disconnected.
\end{Lemma}
\noindent {\it Proof.} Expanding \cite{BraFud2002} where it was shown $P_5$ has no 1-vertex, 2-vertex, 5-vertex and possible $k$-vertices for $P$ with $k \geq 3$ are the sets $A$ = $N_1,_2,_4,_5$, $B$ = $N_1,_3,_4$, and $C$ = $N_2,_3,_5$. Moreover since $G$ is $claw$-free, there are no edges between $A$ and $B$ ($AB$, $A$2, $A$5), between $A$ and $C$ ($AC$, $A$1, $A$4), and between $B$ and $C$ ($BC$, $B$1, $B4$). The sets can have at most one vertex. Let $A$ contains two vertices $x$ and $y$. If $xy \in E$ then $co$-$claw$ ($xy$, $y1$, $1x$, $3$) but if $xy \not\in E$ then $claw$ ($1x$, $1y$, $16$)$-$contradiction. The same argument applies to $B$ and $C$. 3,4-vertex and 0-vertex sets cannot both $!= \phi$ esle there would be a $co$-$claw$. If the set of 3,4-vertex $!= \phi$ then $G$ has at most 8 vertices. If the set of 0-vertex $!= \phi$ then $G$ is a disconnected.


\section{The structure of $k$-critical {\PP}-free
graphs}\label{sec:characterization}
%
In this section, we look at {\CCO}-free graphs that contain a $C_5$ and $co$-$C_l$ for $l \geq 7$. We show that every $k$-critical {\CCO}-free graph $G$ is a clique $K_k$, cycle $C_n$ where $n \geq 5$, a join of a $C_5$ with a critical graph except $C_n$ where $n \geq 5$, a join of a $co$-$cl$ with a critical graph except $C_n$ where $n \geq 5$. 

\begin{Lemma}\label{lem:module-c5critical}
If $G$ contains a $C_5$ then $G$ is $k$-critical for $k \geq 3$ only if $C_5$ is joined on a critical graph except $C_n$ where $n \geq 5$.
%\label{thm:ellcritical}
\end{Lemma}
\noindent {\it Proof.} From {}
\begin{center}
{\bf Acknowledgement}
\end{center}
This work was done by authors  Laurier University.  The authors A.M.H. and C.T.H. D.J.F were each supported by individual NSERC Discovery Grants.


\clearpage
\begin{thebibliography}{99}

\bibitem{BruHoa2009}
    D. Bruce, C. T. Ho\`ang and  J. Sawada,
    A certifying algorithm for 3-colorability of $P_5$-free graphs,
    {\sl Lecture Notes In Computer Science} {\bf 5878} (2009),
    594--604.

\bibitem{ChvHoa1987}
   V. Chv\'atal, C. T. Ho\`ang, N. V. R. Mahadev and D. De Werra,
   Four classes of perfectly orderable graphs,
   {\sl J. Graph Theory} 11:4 (1987), 481--495.

\bibitem{FouGia1995}
    J.-L. Fouquet, V. Giakoumakis, F. Maire and H. Thuillier,
    On graphs without $P_5$ and $\overline{P_5}$,
    {\sl Discrete Math.} 146:1-3 (1995) 33--44.


\bibitem{GiaRus1997}
    V. Giakoumakis and I. Rusu,
    Weighted parameters in $(P_5,\overline{P_5})$-free graphs,
    {\sl Discrete Appl. Math} 80 (1997), 255-�261.

\bibitem{HoaMoo2013}
    C. T. Ho\`ang, B. Moore, D. Recoskie and J. Sawada,
    On $k$-critical $P_5$-free graphs, Proceedings of the VII Latin-American
    Algorithms, Graphs, and Optimization Symposium (LAGOS 2013), {\sl
    Electronic notes in Discrete Mathematics} 44 (2013) 187--193.

\bibitem{HoaKam2008}
    C. T. Ho\`{a}ng, M. Kami\'nski, V. Lozin, J. Sawada and X. Shu,
    A note on $k$-colourability of $P_5$-free graphs,
    {\sl Lecture Notes in Computer Science} 5162 (2008)
    387--394.

\bibitem{HoaKam2010}
    C.T. Ho\`ang, M. Kami\'nski, V.V. Lozin, J.~Sawada and X.~Shu,
    Deciding $k$-colorability of $P_5$-free graphs in polynomial time,
    {\sl Algorithmica} 57:1  (2010) 74--81.

\bibitem{HoaLaz2013}
    C.T. Ho\`ang and D. A. Lazzarato, Polynomial-time algorithms for
    minimum weighted colorings of ($P_5, \overline{P}_5$)-free graphs, manuscript.

\bibitem{Knu1976}
    D. E. Knuth, Mathematics and computer science: coping with
    finiteness, {\sl Science} 194:4271 (1976) 1235--1242.

\bibitem{KraKra2001}
     J. Kratochv\'{\i}l, D. Kr\'al, Zs. Tuza and G.J. Woeginger,
     Complexity of coloring graphs without forbidden induced subgraphs,
     {\sl Lecture Notes in Computer Science} 2204 (2001) 254--262.

\bibitem{MafMor}
    F.~Maffray and G.~Morel.  On $3$-colorable $P_5$-free graphs.  {\sl SIAM
     Journal on Discrete Mathematics} 26 (2012) 1682--1708.

\bibitem{BraFud2002}
  A.~Brandstadt and S.~Mahfud, Maximum Weight Stable Set on graphs without $claw$ and $co$-$claw$ can be solved in linear time, manuscript.

\bibitem{DhaHam2014}
  H. Dhaliwal, A. Hamel, C. Ho\'{a}ng, F. Maffray, T. McConnel and S. Panait, On Color-critical ($P_5, \overline{P}_5$)-free graphs, manuscript.

\end{thebibliography}

\end{document}
