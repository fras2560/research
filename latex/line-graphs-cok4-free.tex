\documentclass[12pt]{article}
\usepackage{latexsym}
\usepackage{tikz}
\usepackage{tkz-graph}
\newcommand*\circled[1]{\tikz[baseline=(char.base)]{
            \node[shape=circle,draw,inner sep=2pt] (char) {#1};}}
\usetikzlibrary{shapes}

\parskip=3pt

\setlength{\textheight}{8.5in}
\setlength{\textwidth}{6in}
\setlength{\topmargin}{0in}
\setlength{\oddsidemargin}{0in}
\setlength{\evensidemargin}{0in}

\newtheorem{Theorem}{Theorem}[section]
\newtheorem{Corollary}[Theorem]{Corollary}
\newtheorem{Lemma}[Theorem]{Lemma}
\newtheorem{Observation}[Theorem]{Observation}

\def\lc{\left\lceil}   
\def\rc{\right\rceil}
\def\inst#1{$^{#1}$}
\def\FAM{($\overline{K_4}$)}

\title{On k-critical {\FAM}-free line graphs}

\author{
	Dallas J. Fraser\inst{1}
	\and Ang\`ele M. Hamel'\inst{1}
	\and Ch\'inh T. Ho\`ang\inst{1}
}
\begin{document}
\maketitle

\begin{center}
{\footnotesize

\inst{1}, Department of Physics and Computer Science, Wilfrid Laurier
University, \\Waterloo, Ontario, Canada}

\end{center}

\begin{abstract}
The coloring of {\FAM}-free $line$ graph is in polynomial.

\noindent{\em Keywords}: Graph coloring, $claw$, $K_5$ - $e$, $\overline{K_4}$, $line$-$graph$ 
\end{abstract}


\section{Introduction}\label{sec:intro}
Deterimining the chromatic number of a graph is a NP-hard problem. For some graphs families the problem can be solved in polynomial time. This paper looks at $\overline{K_4}$ $line$-graphs. For a graph to be a $line$-graph is has no induced subgraph of the following graphs: $claw$, $K_5$ - $e$, $\overline{Twin-C5}$, $\overline{Twin-House}$, etc..

\begin{Lemma}\label{lem:cliqueCutset}
A critical graph that is not a clique does not contain a clique
cutset.
\end{Lemma}

\section{Oberservations}\label{sec:observations}
In this section, we conclude some observations used to prove that there polytime algorithm for {\FAM}-free line-graphs. For the following let $G$ be a line-graph that is {\FAM}-free.

\begin{Lemma}\label{lem:cl-free}
$G$ contains no $C_\ell,\; \ell \geq 8$
\end{Lemma}
\noindent {\it Proof.} Since $G$ is $\overline{K_4}$-free it has no $C_\ell,\; \ell \geq 8$

\begin{Lemma}\label{lem:c7-bounded}
If $G$ contains a $C_7$ then $|G| \leq 14$
\end{Lemma}
\noindent {\it Proof.} Let $G$ contain a $C_7$. Then $G$ has no $k$-vertex for $k \in {0, 1, 2}$ since $G$ is $\overline{K_4}$-free, has no $k$-vertex for $k \in {5, 6,7}$ since $G$ is $line$-free ($claw$), and $G$ has no $k$-vertex for $k \in {3}$ since $G$ is $line$-graph ($\overline{Twin-House}$). $G$ only has a $4$-vertex denoted by $Y_i$ on vertices $i,i+1,i+2,i+4$. Let $y_1$ and $y_2$ be vertices $\in Y_i,\; y_1 \neq y_2$. If $y_1y_2 \not \in E$ then there is a $claw (i+4i+5,i+4y_1,i+4y_2)$ and if $y_1y_2 \in E$ then there is a $\overline{Twin-House}(i,i+1,i+2,y_1,y_2)$ so $|Y_i| \leq 1$. Since $|Y_i| \leq 1$ $G$ can contain at most $14$ vertices.

\begin{Lemma}\label{lem:c5-kvertex}
If $G$ contains $C_5$ then $G$ has no $k$-vertex $\in {1,3,5}$
\end{Lemma}
\noindent {\it Proof.} Let $G$ contain a $C_5$. Then $G$ has no $k$-vertex for $k = 1$ since $G$ is $line$-graph ($claw$), $k = 3$ since $G$ is $line$-graph ($\overline{Twin-C5}$), $k = 5$ since $G$ is $line$-graph ($W_5$). 

For the following let $G$ contain a $C_5$. Let the $0$-vertex set be denoted by $R$, let the $2$-vertex set on vertices $i, i+1$ be denoted by $X_i$, and let the $4$-vertex set on the vertices $i, i+1, i+2, i+3$ be denoted by $Y_i$. Let $X$ denote the set of $2$-vertex and $Y$ denote the set of $4$-vertex.

\begin{Lemma}\label{lem:c5-4vertex-bounded}
$|Y_i| \leq 1$
\end{Lemma}
\noindent {\it Proof.} Let $y_1$ and $y_2$ be vertices $\in Y_i,\; y_1 \neq y_2$. If $y_1y_2 \not \in E$ then there is a $claw (ii-1, iy_1,iy_2)$. If $y_1y_2 \in E$ then there is a $\overline{Twin-House} (i,i+1,i+2,y_1,y_2)$. So $|Y_i| \leq 5$.

\begin{Lemma}\label{lem:yi-cojoin-yi1}
$Y_i \;\circled{0}\; Y_{i+1}$
\end{Lemma}
\noindent {\it Proof.} Let $y_1$ be a vertex from $Y_i$ and $y_2$ be a vertex from $Y_{i+1}$. If $y_1y_2 \in E$ then there is a $K_5 - e (i+1, i+2, i+3, y_1, y_2)$.

\begin{Lemma}\label{lem:yi-join-yi2}
$Y_i \;\circled{1}\; Y_{i+2}$
\end{Lemma}
\noindent {\it Proof.} Let $y_1$ be a vertex from $Y_i$ and $y_2$ be a vertex from $Y_{i+2}$. If $y_1y_2 \in E$ then there is $\overline{A} (i+1, i+2, i+3, i+4, y_1, y_2)$.

\begin{Lemma}\label{lem:xi-clique}
$X_i$ forms a clique
\end{Lemma}
\noindent {\it Proof.} Let $x_1$ and $x_2$ be vertices $\in X_i,\; x_1 \neq x_2$. If $x_1x_2 \not \in E$ then there is a $claw (ii-1, ix_1, ix_2)$. So $x_1x_2 \in E$ and $X_i$ forms a clique.

\begin{Lemma}\label{lem:r-clique}
$R$ forms a clique
\end{Lemma}
\noindent {\it Proof.} Let $r_1$ and $r_2$ be vertices from $R$. If $r_1r_2 \not \in E$ then there is a $\overline{K_4} (0, 2, r_1, r_1)$.

\begin{Lemma}\label{lem:r-join-xi}
$R \; \circled{1} \; X_i$
\end{Lemma}
\noindent {\it Proof.} Let $x$ be a vertex $\in X_i$ and $r$ be a vertex $\in R$. If $rx \not \in E$ then there is a $\overline{K_4} (r, x, i-1, i+2)$.

\begin{Lemma}\label{lem:r-limits-xi}
If $R \neq \phi$ then $|X_i| \leq 2$
\end{Lemma}
\noindent {\it Proof.} Suppose $|X_i| \geq 3$ and $R \neq \phi$. Let $x_1,\; x_2,\;$ and $x_3$ be vertices from $X_i,\; x_1 \neq x_2,\; x_1 \neq x_3,\; x_2 \neq x_3$. By Lemma \ref{lem:xi-clique} $x_1,\; x_2,\;$ and $x_3$ forms a clique. By Lemma \ref{lem:r-join-xi} there is a $K_5$ - $e$ but $G$ is $line$-free.

\begin{Lemma}\label{lem:2xi-cojoins-xj}
A vertex $\in X_i$ cannot have two neighbors in $X_j,\; i \neq j$.
\end{Lemma}
\noindent {\it Proof.} Suppose $|X_j| \geq 2$ with vertices $x_1$ and $x_2$. Let $x_i$ be a vertex from $X_{i}$. If $x_ix_1 \in E$ and $x_ix_2 \in E$ then there is a $\overline{Twin-House}$.

\begin{Lemma}\label{lem:xi-limits-r}
If $X \neq \phi$ then $|R| \leq 2$ or $G$ contains a clique cutset 
\end{Lemma}
\noindent {\it Proof.} If $X_i \neq \phi,\;$ then by Lemma \ref{lem:r-join-xi} $R$ joins $X_i$ and is a cutset. If $X_i \neq \phi,\; X_j \neq \phi,\; i \neq j$. If $|R| \geq 3$ then by Lemma \ref{lem:r-join-xi} and $X_i \; \circled{1} \; X_j$ else there is a $K_5 - e$. Then by Lemma \ref{lem:2xi-cojoins-xj} $|X_i| = 1$ and $|X_j| = 1$. Then $R$ is part of a clique cutset.

\begin{Lemma}\label{lem:r-bounded}
If $R \neq \phi$ then $ |v| \leq 17$ or $G$ contains a clique cutset
\end{Lemma}
\noindent {\it Proof.} If $|R| \geq 3$ then by Lemma \ref{lem:xi-limits-r} is a clique cutset. If $|R| \leq 2$ then by Lemmas \ref{lem:r-limits-xi} and \ref{lem:2xi-cojoins-xj} $|X_i| \leq,\; i \in {0,1,2,3,4}$. By Lemma \ref{lem:one-4vertex} the number of $4$-vertex$ \leq 1$. Then $|V| \leq 2 + 5 + 5 + 5$ or $|R| \geq 3$ and $G$ contains a clique cutset.

\begin{Lemma}\label{lem:r-polynomial}
If $R \neq \phi$ then $G$ can be colored in polynomial time
\end{Lemma}
\noindent {\it Proof.} If $|R| \leq 2$ then by Lemma \ref{lem:r-bounded} is bounded and if $|R| \geq 3$ then by Lemma \ref{lem:cliqueCutset} $G$ cannot be critical and can be colored in polynomial time.

For the following assume $R = \phi$. 

\begin{Lemma}\label{lem:xi-neighbor-adjacent}
$X_i$ cannot have two neighbors $\not \in X_i$ who are non-adjacent
\end{Lemma}
\noindent {\it Proof.} Let $x_i$ denote a vertex $\in X_i$. Let $x_1$ and $x_2$ neighbors of $x_i$. By Lemma \ref{lem:xi-clique} $x_1 \in X_j,\; x_2 \in X_k,\; j \neq k$. If $j=i-1$ and $k=i+1$ then there is a $\overline{A}$ if $x_1x_2 \not in E$. If $j \neq i-1$ and $k \neq j$ then there is a $claw (ix_i, x_1x_i,x_2x_i)$.

\begin{Lemma}\label{lem:one-clique}
If only $X_i \neq \phi,\; i \in ({0, 1, 2, 3, 4})$ then $\omega(G) = \varphi(G)$.
\end{Lemma}
\noindent {\it Proof.} $X_i$, $i$ and $i+1$ forms a clique. Vertex $i+2$ can share a color with a vertex from $X_i$, vertex $i+3$ can share a color with $i$ and vertex $i+4$ can a color with $i+1$.

\begin{Lemma}\label{lem:xi-xi2-clique}
If only $X_i \neq \phi,\; X_{i+1} \neq \phi,\; i \in ({0, 1, 2, 3, 4})$ then $\omega(G) = \varphi(G)$.
\end{Lemma}
\noindent {\it Proof.} $X_i$, $i$, and $i+1$ forms a clique.$|X_i| \geq |X_{i+1}$ then $X_{i+1}$ can be colored with vertices from the clique $X_i$ since by Lemma \ref{lem:2xi-cojoins-xj} each vertex $\in X_{i+1}$ can only be adjacent to one vertex in $\in X_{i}$. Vertex $i+2$ can be colored with same color as vertex $i$, vertex $i+3$ can be colored with the same color as vertex $i+1$, and vertex $i+4$ can be colored with a vertex from $X_i$.

\begin{Lemma}\label{lem:xi-xi3-clique}
If only $X_i \neq \phi,\; X_{i+2} \neq \phi,\; i \in ({0, 1, 2, 3,4})$ then $\omega(G) = \varphi(G)$
\end{Lemma}
\noindent {\it Proof.} If $|X_i| \geq |X_{i+2}|$ then $X_i$, $i$, and $i+1$ forms the largest clique. All of $X_{i+2}$ can be colored with $|X_i|$ colors since by Lemma \ref{lem:2xi-cojoins-xj} each vertex $\in X_{i+1}$ can only be adacent to one vertex $\in X_{i}$. Vertex $i+2$ can be colored with the same color as vertex $i$, vertex $i+3$ can be colored with the same color as vertex $i+1$, and vertex $i+4$ can be colored with a vertex from $X_i$.

For the following assume that at least three consectutive $X_i \neq \phi$.
\begin{Lemma}\label{lem:three-consecutive-xi}
If $|X_j| \geq 3$ then $|X_k| = |X_\ell| = 1,\; j,k,\ell \in ({i,i+1,i+2})$
\end{Lemma}
\noindent {\it Proof.} All vertices $\in X_j \cup X_k \cup X_\ell$ are non-adjacent to vertex $i+4$. The $\chi(X_j \cup X_k \cup X_\ell) \leq 2$ else a $\overline{K_4}$ would form. Let $x_a,_b$ denote a vertex from set $X_a$ where $b \in {1,..., |X_a|}.$. Suppose $|X_j| = 3$ and $|X_k| = |X_\ell| = 1$.  Then $x_k,_1x_\ell,_1 \in E$ otherwise there is a $\overline{K_4}$. Now try to add a vertex to either $X_k$ or $X_\ell$ without loss of generality assume adding to $X_k$. Then $x_j,_1,\;x_k,_2,\;x_\ell,_1$ and $i+4$ forms a $\overline{K_4}$. $x_k,_2x_\ell,_1 \not \in E$ by Lemma \ref{lem:2xi-cojoins-xj}  so $x_j,_1x_\ell,_1 \in E$. Now $x_k,_1x_\ell,_1 \in E$ by Lemma \ref{lem:xi-neighbor-adjacent}. Still $x_j,_2,\;x_k,_2,\;x_\ell,_1$, and $i+4$ forms a $\overline{K_4}$ so $x_j,_2x_k,_2 \in E$. However $x_j,_3,\;x_k,_2,\;x_\ell,_1,\;$, and $i+4$ forms a $\overline{K_4}$ but by Lemma \ref{lem:xi-neighbor-adjacent} $x_j,_3x_k,_2 \not \in E$ and $x_j,_3x_\ell,_1 \not \in E$. Therefor $x_k,_2$ cannont be added.

The following Lemma follows as a result of Lemma \ref{lem:three-consecutive-xi}
\begin{Lemma}\label{lem:f-three-consecutive-xi}
If $|X_j| = 2$ then $|X_k| \leq |X_\ell| \leq 2,\; j,k,\ell \in ({i,i+1,i+2})$
\end{Lemma}

For the following let $x_a$ denote a vertex $\in X_a$
\begin{Lemma}\label{lem:g-bounded}
If $x_ix_{i+1} \in E$, and $x_{i+2}x_{i+3} \in E$ then $G$ is bounded or $x_i \cup x_{i+1} \cup x_{i+2} \cup x_{i+3}$ forms a clique
\end{Lemma}
\noindent {\it Proof.} Suppose $x_ix_{i+1} \in E$ and $x_{i+2}x_{i+3} \in E$. If $x_{i+1}x_{i+2} \not in E$ then there is a $C_7 (i,x_i,x_{i+1}, i+2, x_{i+2}, x_{i+3}, i+4)$ which by Lemma \ref{lem:c7-bounded} $G$ is bounded to at most 14 vertices. If $x_{i+1}x_{i+2} \in E$ then by Lemmas \ref{lem:xi-neighbor-adjacent} and \ref{lem:2xi-cojoins-xj} $x_{i+1}x_{i+3} \in E,\; x_{i}x_{i+3} \in E$ and $x_{i}x_{i+2} \in E$ so now $x_i \cup x_{i+1} \cup x_{i+2} \cup x_{i+3}$ forms a clique.


For the following assume that only $X_i \neq \phi,\; X_{i+2} \neq \phi,\; X_{i+3} \neq \phi$. Let $x_a$ denote a vertex $\in X_a$.
\begin{Lemma}\label{lem:coloring-xi-xi2-xi3}
If only $X_i \neq \phi,\; X_{i+2} \neq \phi,\; X_{i+3} \neq \phi$ then $ \omega(G) = \varphi(G)$
\end{Lemma}
\noindent {\it Proof.} 

Suppose $|X_k| > |X_j| \geq |X_\ell|$ for $k,j,\ell \in ({i,i+2,i+3})$. If $k=i$ then $X_{i+1} \cup X_{i+2}$ can be colored with $|X_i| + 1$ colors since by Lemma \ref{lem:2xi-cojoins-xj} and \ref{lem:xi-neighbor-adjacent}. $X_i \cup i \cup i+1$ forms a clique and $|X_i \cup i \cup i+1| = \omega(G)$. Assume $X_{i+1} \cup X_{i+2}$ used colors $X_i \cup i+1$. Vertex $i+3$ can be colored with vertex $i$. Vertex $i+2$ can use a color from $X_i$since $|X_i| > |X_i+2|$. Vertex $i+4$ can use a color from $X_i$ since $|X_i| > |X_{i+3}|$. If $k=i+2$ then $X_{i} \cup X_{i+3}$ can be colored with $|X_i| + 1$ colors since by Lemma \ref{lem:2xi-cojoins-xj} and \ref{lem:xi-neighbor-adjacent}. $X_{i+2} \cup i+2 \cup i+3$ forms a clique and $|X_i \cup i \cup i+1| = \omega(G)$. Assume $X_{i} \cup X_{i+3}$ uses colors $X_{i+2} \cup i+2$. Vertex $i$ can be colored with vertex $i+3$. Vertex $i+1$ can use a color from $X_{i+2}$ since $|X_{i+2}| > |X_i+2|$. Vertex $i+4$ can use a color from $X_{i+2}$ since $|X_{i+2}| > |X_{i+3}|$. In both cases $G$ was colored with $\omega(G)$ colors so $\omega(G) = \varphi(G)$.

Now suppose $|X_i| = |X_{i+2}| = |X_{i+3}|$ for $k,j,\ell \in ({i,i+2,i+3})$. Starting with the base case when $|X_i| = |X_{i+2}| = |X_{i+3}| = 1$. Worse case $x_i \cup x_{i+2} \cup x_{i+3}$ forms a clique but can color vertex $i$ with color $x_{i+3}$, vertex $i+1$ with color $x_{i+2}$, vertex $i+2$ with color $x_{i+3}$, vertex $i+3$ with color $x_i$ and vertex $i+4$ with color $i+2$. Now adding a vertex $v_1$ to $X_i \cup X_{i+2} \cup X_{i+3}$ will be part of a $K_4$ and will increase $\omega(G)$ by 1. $v_1$ can use the new color added. Without loss of generality assume $v_1$ was added to $X_i$. Adding another vertex to $X_i$ will just continue to add a new color in which that vertex can be assigned the new color. Now add a vertex $v_2$ to $X_{i+2} \cup X_{i+3}$ without loss of generality assume $v_2$ was added to $X_{i+2}$. If $v_1v_2 \not E$ then $v_2$ can be assigned the same color as $v_1$, however if $v_1v_2 \in E$ then $v_2$ can use the same color as $x_i$ or $x_{i+3}$ since by Lemma \ref{lem:2xi-cojoins-xj} $v_2x_i \not \in E$ and $x_{i+3} \not \in E$. Assign $v_2$ the same color as $x_{i+3}$ but $i+2v_2 \in E$ but $i+2$ can use color $v_1$. Now adding a third vertex $v_3$ to $X_{i+3}$. If $v_3v_1 \not \in E$ then by $v_3$ can use the same color as $v_1$. If $v_3v_1 \in E$ then by Lemma \ref{lem:xi-neighbor-adjacent} $v_2v_3 \in E$. $v_3$ can use color $x_{i+2}$ but $v_3i+4 \in E$ however $i+4$ cau use the same color as $v_1$. You can keep adding vertices and this pattern will continue.



\begin{Lemma}\label{lem:xi-xi2-xi3-clique}
If $|X_i| \geq 3$ then $|X_j| = 1$ and 
\end{Lemma}
\noindent {\it Proof.}

\begin{center}
{\bf Acknowledgement}
\end{center}
This work was done by authors  Laurier University. The authors A.M.H. and C.T.H. were each supported by individual NSERC Discovery Grants. D.J.F was supported by an NSERC Undergraduate Student Research Award.


\clearpage
\begin{thebibliography}{99}

\end{thebibliography}

\end{document}
