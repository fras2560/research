\documentclass[12pt]{article}
\usepackage{latexsym}
\usepackage{tikz}
\usepackage{tkz-graph}
\usetikzlibrary{shapes}

\parskip=3pt

\setlength{\textheight}{8.5in}
\setlength{\textwidth}{6in}
\setlength{\topmargin}{0in}
\setlength{\oddsidemargin}{0in}
\setlength{\evensidemargin}{0in}

\newtheorem{Theorem}{Theorem}[section]
\newtheorem{Corollary}[Theorem]{Corollary}
\newtheorem{Lemma}[Theorem]{Lemma}
\newtheorem{Observation}[Theorem]{Observation}

\def\lc{\left\lceil}   
\def\rc{\right\rceil}
\def\inst#1{$^{#1}$}
\def\CK{($claw$, $2K2$)}

\title{On k-critical {\CK}-free graphs}

\author{
	Dallas J. Fraser\inst{1}
	\and Ang\`ele M. Hamel'\inst{1}
	\and Ch\'inh T. Ho\`ang\inst{1}
}
\begin{document}
\maketitle

\begin{center}
{\footnotesize

\inst{1}, Department of Physics and Computer Science, Wilfrid Laurier
University, \\Waterloo, Ontario, Canada}

\end{center}

\begin{abstract}
The coloring of {\CK}-free graph is in polynomial.

\noindent{\em Keywords}: Graph coloring, $claw$, $2K2$
\end{abstract}


\section{Introduction}\label{sec:intro}

\begin{Theorem}\label{thm:ben-rebea}
Let $G$ be a connected $claw$-free graph with $\alpha(G) \geq 3$. If $G$ contains an odd anti-hole then it contains an odd antihole then it contains a $C_5$ $\Box$
\end{Theorem}

\begin{Theorem}\label{thm:alpha-two-poly}
There is a polytime algorithm for coloring a $G$ with $\alpha(G) =2$.
 \end{Theorem}

\section{Oberservations}\label{sec:observations}
In this section, we conclude some observations used to prove that there polytime algorithm for ${\CK}$-free graphs.
\begin{Lemma}\label{lem:odd-hole-free}
$G$ cannot contain an induced $C_\ell$, $\ell \geq 7$.
\end{Lemma}
\noindent {\it Proof}. Let $G$ be a $C_\ell$, $\ell \geq 7$ with vertices 1, 2, ...,$\ell_{-1},\; \ell$. This forms a $2K2$ with (2,3) and ($\ell_{-1},\; \ell$) but $G$ is {\CK}-free. $\Box$

\begin{Lemma}\label{lem:alpha-2-c5}
If $G$ contains an $C_5$ then $\alpha(G) = 2$ or $G$ is disconnnected
\end{Lemma}
\noindent{\it Proof}. 
$G$ contains no $k$-vertex for $k \in {1, 2, 3}$. There can be no edge between an $0$-vertex and a $4$-vertex or a $5$-vertex. Let $x$ be a $0$-vertex and $y$ be a $4$-vertex, if $xy \in E$ then $G$ contains a $claw (xy, yi, yi_{+2})$. Similar argument applies to $5$-vertex. $\alpha(G) =2$ if $G$ contains no $0$-vertex since $\alpha(C_5) = 2$ and  all $4$-vertex and $5$-vertex must have $|E| = v -2$ otherwise there is a $claw$.

\begin{Lemma}\label{lem:alpha-2-anti-hole}
If $G$ contains an odd anti-hole then $\alpha(G) =2$ or $G$ is disconnected.
\end{Lemma}
\noindent{\it Proof}.
By Theorem \ref{thm:ben-rebea} if connected $G$ contains an odd anti-hole with $\alpha(G) \geq 3$ then $G$ contains an induced $C_5$ but by Lemma \ref{alpha-2-c5} any $G$ containing a $C_5$ cannot have $\alpha(G) != 2$ if it is connected. So therefore if connected $G$ contains an odd anti-hole $\alpha(G) = 2$.

\begin{Theorem}\label{thm:poly-colorable}
If {\CK}-free then $G$ can be colored in polynomial time
\end{Theorem}

By Lemmas \ref{alpha-2-anti-hole} and \ref{alpha-2-c5} any $G$ contains an odd anti-hole has $\alpha(G) = 2$ and using \ref{alpha-2-poly} can be colored in polynomial time. If $G$ does not contain an odd-anti hole and by Lemma \ref{odd-hole-free} if is a perfect graph. A perfect graph can be colored in polynomial time.

\begin{center}
{\bf Acknowledgement}
\end{center}
This work was done by authors  Laurier University. The authors A.M.H. and C.T.H. were each supported by individual NSERC Discovery Grants. D.J.F was supported by an NSERC Undergraduate Student Research Award.


\clearpage
\begin{thebibliography}{99}

\end{thebibliography}

\end{document}
