\documentclass[12pt]{article}
\usepackage{latexsym}
\usepackage{tikz}
\usepackage{tkz-graph}
\newcommand*\circled[1]{\tikz[baseline=(char.base)]{
            \node[shape=circle,draw,inner sep=2pt] (char) {#1};}}
\usetikzlibrary{shapes}

\parskip=3pt

\setlength{\textheight}{8.5in}
\setlength{\textwidth}{6in}
\setlength{\topmargin}{0in}
\setlength{\oddsidemargin}{0in}
\setlength{\evensidemargin}{0in}

\newtheorem{Theorem}{Theorem}[section]
\newtheorem{Corollary}[Theorem]{Corollary}
\newtheorem{Lemma}[Theorem]{Lemma}
\newtheorem{Observation}[Theorem]{Observation}

\def\lc{\left\lceil}   
\def\rc{\right\rceil}
\def\inst#1{$^{#1}$}
\def\KCD{($K4$, $co-diamond$)}
\def\VTWO{$N_{i , i+1}$}
\def\VJTWO{$N_{j , j+1}$}
\def\VTHREE{$N_{i , i+1, i+2}$}
\def\VFOUR{$N_{i , i+1, i+2, i+3}$}

\title{Observations on {\KCD}-free graphs}

\author{
	Dallas J. Fraser\inst{1}
	\and Ang\`ele M. Hamel'\inst{1}
	\and Ch\'inh T. Ho\`ang\inst{1}
}
\begin{document}
\maketitle

\begin{center}
{\footnotesize

\inst{1}, Department of Physics and Computer Science, Wilfrid Laurier
University, \\Waterloo, Ontario, Canada}

\end{center}

\begin{abstract}
Just some observations I have concluded on {\KCD}-free graphs.

\noindent{\em Keywords}: Graph coloring, $claw$, $2K2$
\end{abstract}


\section{Introduction}\label{sec:intro}

\begin{Theorem}\label{thm:ben-rebea}
Let $G$ be a connected $claw$-free graph with $\alpha(G) \geq 3$. If $G$ contains an odd anti-hole then it contains an odd antihole then it contains a $C_5$ $\Box$
\end{Theorem}

\section{Oberservations}\label{sec:observations}
In this section, we conclude some observations about {\KCD}-free graphs. Assume $G$ is a {\KCD}-free graph.
\begin{Observation}\label{obs:no-cl}
$G$ has no $C_\ell$, $\ell \geq 7$.
\end{Observation}
\noindent {\it Proof.} Let $G$ be a $C_\ell$, $\ell \geq 7$ with vertices 1, 2, ...,$\ell_{-1},\; \ell$. This forms a $co$-$diamond$ with $(1-2, 4, \ell_{-1},\; )$.

\begin{Observation}\label{obs:no-anti-cl}
$G$ has no $\overline{C_\ell}$, $\ell > 7$.
\end{Observation}
\noindent {\it Proof.} Let $G$ be a $\overline{C_\ell}$, $\ell > 7$ with vertices 1, 2, ...,$\ell_{-1},\; \ell$. This forms a $K_4$ with $(1, 3, 5, 7)$.

\begin{Observation}\label{obs:c7-has-c5}
If $G$ contains a $C_7$ then $G$ is a $C_7$ or contains a $C_5$ 
\end{Observation}
\noindent {\it Proof.} Let $G$ be a contain a $\overline{C_7}$, with vertices $1, 2, ...,7$. Then $G$ has no $0$-vertex, $1$-vertex, and $2$-vertex because $G$ is $co$-$diamond$-free and $G$ has no $5$-vertex, $6$-vertex, and $7$-vertex because $G$ is $K_4$-free. Adding a $3$-vertex or $4$-vertex makes $\alpha(G) = 3$ and by Theorem \ref{thm:ben-rebea} it has a $C_5$

For the following assume $G$ contains $C_5$ with vertices $0, 1, ..,4$ forming the $C_5$. Let $X_i$ denote the set of $2$-vertices for $C$ adjacent to $i, i+1$, $Y_i$ denote the set of $3$-vertices for $C_5$ adjacent to $i-1, i, i+1$, $W_i$ denote the set of $3$-vertices for $C_5$ adjacent to $i, i+2, i+3$, let $Z_i$ denote the set of $4$-vertices for $C_5$ adjacent to $i-1,i,i+1,i+2$  and $U$ denote the set of $5$-vertices for $C_5$.

\begin{Observation}\label{obs:Xi-is-one}
$|X_i| = 1$
\end{Observation}
\noindent {\it Proof.} Let $x$ and $y$ be vertices from $X_i$. If $xy \not \in E$ then there is a $co$-$diamond (x,y,i+2i+3)$ but if $xy \in E$ there is a $K_4 (i, i+1, x, y)$ but $G$ is {\KCD}-free.

\begin{Observation}\label{obs:Xi-clique}
$N_i \;\circled{1}\; N_0 \cup N_1 \cup N_2 \cup N_3 \cup N_4$
\end{Observation} 
\noindent {\it Proof.} Let $x$ be a vertex from $X_i$ and $y$ a vertex from $N_j$. By Observation \ref{obs:Xi-is-one} if $i = j$ then $x =y$. If $j = i + 1$ and $xy \not \in E$ then there is a $co$-$diamond (x, y, i+3i+4)$. If $ j =i +2$ and $xy \not \in E$ then there is a $co$-$diamond (xi, y, i-1)$.

\begin{Observation}\label{obs:Yi-is-one}
$|Y_i| = 1$
\end{Observation} 
\noindent {\it Proof.} Let $x$ and $y$ be a vertex from $Y_i$. If $xy \not \in E$ then there is $co$-$diamond(x,y, i+2i+3)$ and if $xy \in E$ there is a $K_4 (x, y, i, i+1)$ but $G$ is {\KCD}-free.

\begin{Observation}\label{obs:Yi-co-joins}
$Y_i \;\circled{0}\; Y_{i-1} \cup Y_i \cup Y_{i+1}$
\end{Observation}
\noindent {\it Proof.} Let $x$ be a vertex from $Y_i$ and $y$ be a vertex $from Y_{i-1} \cup Y_i \cup Y_{i+1}$. If $xy \in E$ then there is a $K_4 (x, y, i, i+1)$ or $(x, y, i, i-1)$ but $G$ is {\KCD}-free.

\begin{Observation}\label{obs:Wi-stable}
The set $W_i$ forms a stable set
\end{Observation}
\noindent {\it Proof.} Let $x$ and $y$ be vertices from $W_i$. If $xy \in E$ then there is a $K_4 (x, y, i+2, i+3)$.

\begin{Observation}\label{obs:U-stable}
The $U$ forms a stable set
\end{Observation}
\noindent {\it Proof.} Let $x$ and $y$ be vertices from $U$. If $xy \in E$ then there is a $K_4 (x, y, 0, 1)$.

\begin{Observation}\label{obs:U-co-joins}
$U \;\circled{0}\; X_i \cup Y_i \cup Z_i \cup W_i$
\end{Observation}
\noindent {\it Proof.} Let $x$ be a vertex from $U$ and $y$ be a vertex from $X_i \cup Y_i \cup Z_i \cup W_i$. If $xy \in E$ then there is $K_4$ since all vertices from $X_i \cup Y_i \cup Z_i \cup W_i$ see two consective vertices in $C_5$.


\begin{center}
{\bf Acknowledgement}
\end{center}
This work was done by authors  Laurier University. The authors A.M.H. and C.T.H. were each supported by individual NSERC Discovery Grants. D.J.F was supported by an NSERC Undergraduate Student Research Award.


\clearpage
\begin{thebibliography}{99}

\end{thebibliography}

\end{document}
